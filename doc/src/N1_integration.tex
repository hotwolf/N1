%###############################################################################
%# N1 - Manual - Integration Guide                                             #
%###############################################################################
%#    Copyright 2018 - 2019 Dirk Heisswolf                                     #
%#    This file is part of the N1 project.                                     #
%#                                                                             #
%#    N1 is free software: you can redistribute it and/or modify               #
%#    it under the terms of the GNU General Public License as published by     #
%#    the Free Software Foundation, either version 3 of the License, or        #
%#    (at your option) any later version.                                      #
%#                                                                             #
%#    N1 is distributed in the hope that it will be useful,                    #
%#    but WITHOUT ANY WARRANTY; without even the implied warranty of           #
%#    MERCHANTABILITY or FITNESS FOR A PARTICULAR PURPOSE.  See the            #
%#    GNU General Public License for more details.                             #
%#                                                                             #
%#    You should have received a copy of the GNU General Public License        #
%#    along with N1.  If not, see <http://www.gnu.org/licenses/>.              #
%###############################################################################
%# Version History:                                                            #
%#   March 4, 2019                                                             #
%#      - Initial release                                                      #
%###############################################################################

\section{Integration Guide}
\label{integration}

This section outlines the interfaces and configurations of the N1 processor
for system integration.

\subsection{Integratation Parameters}
\label{integration:params}

The N1 processor supports six \gls{verilog} integration parameters to configure
the design for application specific needs:

\begin{description}[style=nextline]

\item[\texttt{SP\_WIDTH}] Stack pointer width. \\
  This parameter determines the address width of the \gls{ls}.
  Values in the range of 5 to 16 are valid.
  The default value is 12. 
  
\item[\texttt{IPS\_DEPTH}] Depth of the intermediate parameter stack. \\
  This parameter determines the number of \glspl{cell} in the \gls{is} of the \gls{ps}.
  Any value larger than 2 is valid.
  The default value is 8. 
  The purpose of the \gls{is} is to conceal fluctuations in stack usage to the \gls{ls}.
  The optimal value should be derived from the application use case.

\item[\texttt{IRS\_DEPTH}] Depth of the intermediate return stack. \\
  This parameter determines the number of \glspl{cell} in the \gls{is} of the \gls{rs}.
  Any value larger than 2 is valid.
  The default value is 8. 
  The purpose of the \gls{is} is to conceal fluctuations in stack usage to the \gls{ls}.
  The optimal value should be derived from the application use case.

\item[\texttt{PBUS\_AADR\_OFFSET}] Offset for direct \gls{jump} or \gls{call} addressing. \\
  This parameter determines the location of the 32KB window for \glspl{jump} and \glspl{call}
  with \gls{diradr}.
  The default value is \texttt{0x0000}. 

\item[\texttt{PBUS\_MADR\_OFFSET}] Offset for direct data accesses. \\
  This parameter determines the location of the 511B window for memory I/O with \gls{diradr}.
  This window should cover commonly used \gls{forth} variables. The default value is \texttt{0xFFFF}. 

\item[\texttt{PS\_RS\_DIST}] Safety distance between the \gls{ps} and the \gls{rs}. \\
  Recovering from an exception requires some free \gls{stack} space.
  This parameter determines the remaining \gls{stack} space when a \gls{stack} overflow exception is thrown.
  The default value is 22. The optimal value depends on the requirements of the exception handler software.

\end{description}

\subsection{Interfaces}
\label{integration:if}

The N1 processor provides four interfaces which must be connected at system level.
A fifth one (see \secref{integration:if:prb}) is only to be used for verification and debug purposes.

\subsubsection{Clock and Resets}
\label{integration:if:clk}
This interface provides clocks and resets for all sequential logic in the N1 design.

\begin{description}[style=nextline]

\item[\texttt{clk\_i}] Single clock input. \\  
  This clock is used for all interfaces as well as all internal sequential logic.

\item[\texttt{async\_rst\_i}] Asynchronous reset input. \\
  This active high reset input may assert asynchronously, but must deassert synchronously.
  This signal is not required if a synchrounous reset (\texttt{sync\_rst\_i}) is implemented.
  If unused, this input must be tied to \texttt{0}.

\item[\texttt{sync\_rst\_i}] Synchronous reset input. \\
  This active high reset input must assert and deassert synchronously.
  This signal is not required if an asynchrounous reset (\texttt{async\_rst\_i}) is implemented.
  If unused, this input must be tied to \texttt{0}.
 
\end{description}

\subsubsection{Program Bus}
\label{integration:if:pbus}
This interface connects the N1 to the main memory.
All signals comply to the \gls{wb} protocoll~\cite{wishbone}.

\begin{description}[style=nextline]
  
\item[\texttt{pbus\_cyc\_o}] Cycle indicator output. \\
  This output signal corresponds to signal \texttt{CYC\_O} of the Wishbone specification~\cite{wishbone}.

\item[\texttt{pbus\_stb\_o}] Strobe output. \\   
  This output signal corresponds to signal \texttt{STB\_O} of the Wishbone specification~\cite{wishbone}.

\item[\texttt{pbus\_we\_o}]  Write enable output. \\
  This output signal corresponds to signal \texttt{WE\_O} of the Wishbone specification~\cite{wishbone}.

\item[\texttt{pbus\_adr\_o}] Address bus. \\   
  These output signals correspond to bus \texttt{ADR\_O} of the Wishbone specification~\cite{wishbone}.

\item[\texttt{pbus\_dat\_o}] Write data bus. \\    
  These output signals correspond to bus \texttt{DAT\_O} of the Wishbone specification~\cite{wishbone}.

\item[\texttt{pbus\_tga\_cof\_jmp\_o}] Change of flow indicator. \\   
  This output signal corresponds to bus \texttt{TGA\_O} of the Wishbone specification~\cite{wishbone}.
  It indicates, that the current bus access was caused by a \gls{jump} instruction.
  This information may be used to trace the program flow.

\item[\texttt{pbus\_tga\_cof\_cal\_o}] Change of flow indicator. \\   
  This output signal corresponds to bus \texttt{TGA\_O} of the Wishbone specification~\cite{wishbone}.
  It indicates, that the current bus access was caused by either a \gls{call} instruction or an
  interrupt service request.
  This information may be used to trace the program flow.

\item[\texttt{pbus\_tga\_cof\_bra\_o}] Change of flow indicator. \\   
  This output signal corresponds to bus \texttt{TGA\_O} of the Wishbone specification~\cite{wishbone}.
  It indicates, that the current bus access was caused by a \gls{branch} instruction.
  This information may be used to trace the program flow.

\item[\texttt{pbus\_tga\_cof\_eow\_o}] Change of flow indicator. \\   
  This output signal corresponds to bus \texttt{TGA\_O} of the Wishbone specification~\cite{wishbone}.
  It indicates ,that the current bus access was caused by a return from a \gls{call}.
  This information may be used to trace the program flow.

\item[\texttt{pbus\_ack\_i}] Acknowlede input. \\   
  This input signal corresponds to signal \texttt{ACK\_I} of the Wishbone specification~\cite{wishbone}.
  If unused, this input must be tied to \texttt{1}.

\item[\texttt{pbus\_err\_i}] Error indicator input. \\  
  This input signal corresponds to signal \texttt{ERR\_I} of the Wishbone specification~\cite{wishbone}.
  It informs the N1 processor, that the current address exceeds the valid range of the connected
  memory system. 
  If unused, this input must be tied to \texttt{0}.
  
\item[\texttt{pbus\_stall\_i}] Pipeline stall input. \\
  This input signal corresponds to signal \texttt{STALL\_I} of the Wishbone specification~\cite{wishbone}.
  If unused, this input must be tied to \texttt{0}.

\item[\texttt{pbus\_dat\_i}] Read data bus. \\ 
  These input signals correspond to bus \texttt{DAT\_I} of the Wishbone specification~\cite{wishbone}.

\end{description}

\subsubsection{Stack Bus}
\label{integration:if:sbus}
This interface connects the N1 to the stack memory.
It is expected that the \texttt{SP\_WIDTH} parameter (see \secref{integration:params}) matches the
implemented memory size. Therefore no \texttt{ERR\_I} input is needed in this interface.
All signals comply to the \gls{wb} protocoll~\cite{wishbone}.

\begin{description}[style=nextline]

\item[\texttt{sbus\_cyc\_o}] Cycle indicator output. \\
  This output signal corresponds to signal \texttt{CYC\_O} of the Wishbone specification~\cite{wishbone}.

\item[\texttt{sbus\_stb\_o}] Strobe output. \\   
  This output signal corresponds to signal \texttt{STB\_O} of the Wishbone specification~\cite{wishbone}.

\item[\texttt{sbus\_we\_o}]  Write enable output. \\
  This output signal corresponds to signal \texttt{WE\_O} of the Wishbone specification~\cite{wishbone}.

\item[\texttt{sbus\_adr\_o}] Address bus. \\   
  These output signals correspond to bus \texttt{ADR\_O} of the Wishbone specification~\cite{wishbone}.

\item[\texttt{sbus\_dat\_o}] Write data bus. \\    
  These output signals correspond to bus \texttt{DAT\_O} of the Wishbone specification~\cite{wishbone}.

\item[\texttt{sbus\_tga\_ps\_o}] \Gls{ps} access indicator. \\   
  These output signals correspond to bus \texttt{TGA\_O} of the Wishbone specification~\cite{wishbone}.
  It indicates, that the current bus access is associated with the \gls{ps}.

\item[\texttt{sbus\_tga\_rs\_o}] \Gls{rs} access indicator. \\   
  These output signals correspond to bus \texttt{TGA\_O} of the Wishbone specification~\cite{wishbone}.
  It indicates, that the current bus access is associated with the \gls{rs}.
  
\item[\texttt{sbus\_ack\_i}] Acknowlede input. \\   
  This input signal corresponds to signal \texttt{ACK\_I} of the Wishbone specification~\cite{wishbone}.
  If unused, this input must be tied to \texttt{1}.

\item[\texttt{sbus\_stall\_i}] Pipeline stall input. \\
  This input signal corresponds to signal \texttt{STALL\_I} of the Wishbone specification~\cite{wishbone}.
  If unused, this input must be tied to \texttt{0}.

\item[\texttt{sbus\_dat\_i}] Read data bus. \\ 
  These input signals correspond to bus \texttt{DAT\_I} of the Wishbone specification~\cite{wishbone}.

\end{description}

\subsubsection{Interrupt Interface}
\label{integration:if:irq}
This interface connects an optional interrupt controller to the N1 processor. 


\begin{description}[style=nextline]

\item[\texttt{irq\_ack\_o}] Interrupt acknowledge. \\
  This output signal asserts for one clock cycle, whenever the current interrupt is serviced.
  It may be used for automatic flag clearing.

\item[\texttt{irq\_req\_i}] Interrupt request. \\
  Any non-zero value driven to this bus interface is interpreted as interrupt request.
  The value determines the start address of the interrupt service routine that is to be executed by the
  N1 processor. This bus must be tied to \texttt{0x0000} if no interrupt controller is connected.
  
\end{description}

\subsubsection{Probe Signals}
\label{integration:if:prb}
This interface propagates all internal states of the N1 processor to the outside.
It is solely intended for verification and debug purposes and should be left unconnected for system integration.
The signals in this interface are specific to the internal implementation of the N1 processor and may change
with every revision.

\subsection{Target Specific Design Files}
\label{integration:ifs}
All adder and multiplier logic of the N1 design ls located in a single \gls{verilog} module called \texttt{N1\_dsp}.
A synthesizable implementation of this module, can be found in the file \texttt{rtl/verolog/N1\_dsp\_synth.v}.
If desired, this file can be replaced by one containing a alternative implementation of the \texttt{N1\_dsp} module.
An example is given in in the file \texttt{rtl/verolog/N1\_dsp\_iCE40UP5K.v}.
It contains a custom implementation for Lattice iCE40 FPGAs, utilizing four hard instantiated \texttt{SB\_MAC16}
macro cells.

