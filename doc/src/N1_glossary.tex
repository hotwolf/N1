%###############################################################################
%# N1 - Manual - Glossary                                                      #
%###############################################################################
%#    Copyright 2018 Dirk Heisswolf                                            #
%#    This file is part of the N1 project.                                     #
%#                                                                             #
%#    N1 is free software: you can redistribute it and/or modify               #
%#    it under the terms of the GNU General Public License as published by     #
%#    the Free Software Foundation, either version 3 of the License, or        #
%#    (at your option) any later version.                                      #
%#                                                                             #
%#    N1 is distributed in the hope that it will be useful,                    #
%#    but WITHOUT ANY WARRANTY; without even the implied warranty of           #
%#    MERCHANTABILITY or FITNESS FOR A PARTICULAR PURPOSE.  See the            #
%#    GNU General Public License for more details.                             #
%#                                                                             #
%#    You should have received a copy of the GNU General Public License        #
%#    along with N1.  If not, see <http:%www.gnu.org/licenses/>.               #
%###############################################################################
%# Version History:                                                            #
%#   Novemmber 27, 2018                                                        #
%#      - Initial release                                                      #
%###############################################################################

%Glossary 
\newglossaryentry{ram} {
    name={RAM},
    description={
      Random access memory.
      \nopostdesc
    }
}

\newglossaryentry{forth} {
    name={Forth},
    description={
      Forth is a extensible stack-based programming language.
      \nopostdesc
    }
}
 
\newglossaryentry{byte} {
    name={byte},
    description={
      An 8-bit data entity.
      \nopostdesc
    }
}

\newglossaryentry{word} {
    name={word},
    description={
      The term word  refers to a callable code sequence in \Gls{forth} terminology.
      \nopostdesc
    }
}

\newglossaryentry{jump} {
    name={jump},
    description={
      A change of the program flow without return option (see \secref{opcodes:jump}).
      \nopostdesc
    }
}

\newglossaryentry{branch} {
    name={conditional branch},
    description={
      A change of the program flow without return option, only if a certain (non-zero)
      argument value is given (see \secref{opcodes:branch}).
      \nopostdesc
    },
    plural={conditional branches}
}

\newglossaryentry{call} {
    name={call},
    description={
      A change of the program flow, where a return address is kept
      on the \gls{rs} (see \secref{opcodes:call}).
      \nopostdesc
    }
}

\newglossaryentry{literal} {
    name={literal},
    description={
      A fixed numerical value within the program code (see \secref{opcodes:literal}).
      \nopostdesc
    }
}

\newglossaryentry{semicolon} {
    name={;},
    description={
      End of a \gls{word} definition in \Gls{forth}.
      \nopostdesc
    }
}

\newglossaryentry{stack} {
    name={stack},
    description={
      A \gls{lifo} storage.
      \nopostdesc
    }
}

\newglossaryentry{cell} {
    name={cell},
    description={
      A data entity within a \gls{stack}.
      \nopostdesc
    }
}

\newglossaryentry{tos} {
    name={TOS},
    description={
      The top \gls{cell} of a \gls{stack}.
      \nopostdesc
    }
}

\newglossaryentry{rs} {
    name={return stack},
    description={
      A \gls{lifo} storage mainly for maintaining return addresses
      of \glspl{call}.
      \nopostdesc
    }
}

\newglossaryentry{ps} {
    name={parameter stack},
    description={
      A \gls{lifo} storage mainly for keeping call parameters and
      return values.
      \nopostdesc
    }
}

\newglossaryentry{lifo} {
    name={LIFO},
    description={
      A memory which is accessible in last in - first out order.
      \nopostdesc
    }
}

\newglossaryentry{opcode} {
    name={opcode},
    description={
      Encoding of a machine instruction. Short for ``operation code''.
      \nopostdesc
    }
}

\newglossaryentry{alu} {
    name={ALU},
    description={
      Arithmetic Logic Unit.
      \nopostdesc
    }
}

\newglossaryentry{ust} {
    name={UST},
    description={
      A bit field in the stack instruction which contols data movement
      between two neighboring \glspl{cell} in the upper \gls{ps} or \gls{rs}.
      The mnemonic stands for ``\textbf{U}pper \textbf{S}tack \textbf{T}ransition''.
      \nopostdesc
    }
}

\newglossaryentry{ist} {
    name={IST},
    description={
      A bit field in the stack instruction which contols data movement
      on the intermediate \gls{ps} or \gls{rs}.
      The mnemonic stands for
      ``\textbf{I}intermediate \textbf{S}tack \textbf{T}ransition''.
      \nopostdesc
    }
}

\newglossaryentry{ls} {
    name={lower stack},
    description={
      The section of the stack which stored in RAM.
      See \secref{stacks}. 
      \nopostdesc
    }
}

\newglossaryentry{is} {
    name={intermediate stack},
    description={
      The section of the stack, which serves as a buffer between the
      \gls{ls} and the \gls{us}.
      See \secref{stacks}. 
      \nopostdesc
    }
}

\newglossaryentry{us} {
    name={upper stack},
    description={
      The section of the stack, which contains the \gls{tos}. It supports
      reordering of its storage \gls{cell}.
      See \secref{stacks}. 
      \nopostdesc
    }
}

\newglossaryentry{msb} {
    name={MSB},
    description={
      The most significant bit.
      \nopostdesc
    }
}

\newglossaryentry{lsb} {
    name={LSB},
    description={
      The least significant bit.
      \nopostdesc
    }
}

\newglossaryentry{tc} {
    name={throw code},
    description={
      A unique identifier for each type of exception.
      \nopostdesc
    }
}


