%###############################################################################
%# N1 - Manual - Overview                                                      #
%###############################################################################
%#    Copyright 2018 Dirk Heisswolf                                            #
%#    This file is part of the N1 project.                                     #
%#                                                                             #
%#    N1 is free software: you can redistribute it and/or modify               #
%#    it under the terms of the GNU General Public License as published by     #
%#    the Free Software Foundation, either version 3 of the License, or        #
%#    (at your option) any later version.                                      #
%#                                                                             #
%#    N1 is distributed in the hope that it will be useful,                    #
%#    but WITHOUT ANY WARRANTY; without even the implied warranty of           #
%#    MERCHANTABILITY or FITNESS FOR A PARTICULAR PURPOSE.  See the            #
%#    GNU General Public License for more details.                             #
%#                                                                             #
%#    You should have received a copy of the GNU General Public License        #
%#    along with N1.  If not, see <http:%www.gnu.org/licenses/>.               #
%###############################################################################
%# Version History:                                                            #
%#   Novemmber 26, 2018                                                        #
%#      - Initial release                                                      #
%###############################################################################

\section{Overview}
\label{overview}

The N1 is a snall stack machine, inspired by the J1 Forth CPU\cite{j1}. 
Just like its paragon, the N1 is a 16-bit processor wich implements basic Forth words directly
in hardware. However the N1 parts from the J1's simplistic design approach in in two ways:

\begin{itemize}
  \item The N1 support a larger code space of up to 32KB. Therefore it has its own instruction set
  \item The N1 implements its parameter and return stacks as shallow register stacks, which overflow
        into RAM. The overall depth of each stack is determined by the available RAM. 
\end{itemize}


