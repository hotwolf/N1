%###############################################################################
%# N1 - Manual - Glossary                                                      #
%###############################################################################
%#    Copyright 2018 Dirk Heisswolf                                            #
%#    This file is part of the N1 project.                                     #
%#                                                                             #
%#    N1 is free software: you can redistribute it and/or modify               #
%#    it under the terms of the GNU General Public License as published by     #
%#    the Free Software Foundation, either version 3 of the License, or        #
%#    (at your option) any later version.                                      #
%#                                                                             #
%#    N1 is distributed in the hope that it will be useful,                    #
%#    but WITHOUT ANY WARRANTY; without even the implied warranty of           #
%#    MERCHANTABILITY or FITNESS FOR A PARTICULAR PURPOSE.  See the            #
%#    GNU General Public License for more details.                             #
%#                                                                             #
%#    You should have received a copy of the GNU General Public License        #
%#    along with N1.  If not, see <http:%www.gnu.org/licenses/>.               #
%###############################################################################
%# Version History:                                                            #
%#   Novemmber 27, 2018                                                        #
%#      - Initial release                                                      #
%###############################################################################

\newglossaryentry{forth} {
    name={Forth},
    description={
      Forth is a extensible stack-based programming language.
      \nopostdesc
    }
}
 
\newglossaryentry{byte} {
    name={byte},
    description={
      An 8-bit data entity.
      \nopostdesc
    }
}

\newglossaryentry{word} {
    name={word},
    description={
      The term word is used in two different contexts throughout
      this document. It refers to either a 16-bit data entity or
      a callable code sequence in \Gls{forth} terminology.
      \nopostdesc
    }
}

\newglossaryentry{jump} {
    name={jump},
    description={
      A change of the program flow without return option.
      \nopostdesc
    }
}

\newglossaryentry{branch} {
    name={branch},
    description={
      A change of the program flow without return option, where the
      destination is given as offset from the start point.
      \nopostdesc
    },
    plural={Branches}
}

\newglossaryentry{call} {
    name={call},
    description={
      A change of the program flow, where a return address is kept
      on the \gls{rs}.
      \nopostdesc
    }
}

\newglossaryentry{literal} {
    name={literal},
    description={
      A fixed numerical value within the program code.
      \nopostdesc
    }
}

\newglossaryentry{semicolon} {
    name={;},
    description={
      End of a \gls{word} definition in \Gls{forth}.
      \nopostdesc
    }
}

\newglossaryentry{rs} {
    name={return stack},
    description={
      A \gls{lifo} storage mainly for maintaining return addresses
      of \glspl{call}.
      \nopostdesc
    }
}

\newglossaryentry{ps} {
    name={parameter stack},
    description={
      A \gls{lifo} storage mainly for keeping call parameters and
      return values.
      \nopostdesc
    }
}

\newglossaryentry{lifo} {
    name={LIFO},
    description={
      A memory which is accessible in last in - first out order.
      \nopostdesc
    }
}

\newglossaryentry{opcode} {
    name={opcode},
    description={
      Encoding of a machine instruction. Short for ``operation code''.
      \nopostdesc
    }
}

\newglossaryentry{alu} {
    name={ALU},
    description={
      Arithmetic Logic Unit.
      \nopostdesc
    }
}
