%###############################################################################
%# N1 - Manual - Integration Guide                                             #
%###############################################################################
%#    Copyright 2018 - 2019 Dirk Heisswolf                                     #
%#    This file is part of the N1 project.                                     #
%#                                                                             #
%#    N1 is free software: you can redistribute it and/or modify               #
%#    it under the terms of the GNU General Public License as published by     #
%#    the Free Software Foundation, either version 3 of the License, or        #
%#    (at your option) any later version.                                      #
%#                                                                             #
%#    N1 is distributed in the hope that it will be useful,                    #
%#    but WITHOUT ANY WARRANTY; without even the implied warranty of           #
%#    MERCHANTABILITY or FITNESS FOR A PARTICULAR PURPOSE.  See the            #
%#    GNU General Public License for more details.                             #
%#                                                                             #
%#    You should have received a copy of the GNU General Public License        #
%#    along with N1.  If not, see <http://www.gnu.org/licenses/>.              #
%###############################################################################
%# Version History:                                                            #
%#   March 4, 2019                                                             #
%#      - Initial release                                                      #
%###############################################################################

\section{Integration Guide}
\label{integration}

This section explains the interfaces and configurations of the N1 processor
for system integration.

\subsection{Integratation Parameters}
\label{integration:params}

The N1 processor supports six integration parameters to configure the design for
application specific needs:

\begin{description}[style=nextline]

\item[\texttt{SP\_WIDTH}] Stack pointer width. \\

\item[\texttt{IPS\_DEPTH}] Depth of the intermediate parameter stack. \\

\item[\texttt{IRS\_DEPTH}] Depth of the intermediate return stack. \\

\item[\texttt{PBUS\_AADR\_OFFSET}] Offset for direct \gls{jump} or \gls{call} addressing. \\

\item[\texttt{PBUS\_MADR\_OFFSET}] Offset for direct data accesses. \\

\item[\texttt{PS\_RS\_DIST}] Safety distance between the \gls{ps} and the \gls{rs}. \\



  
\end{description}


\subsection{Interfaces}
\label{integration:if}

\subsubsection{Clock and Resets}
\label{integration:if:clk}

\begin{description}[style=nextline]

\item[\texttt{clk\_i}] Common clock input for all Wishbone interfaces.\\  
  This clock input  corresponds to signal \texttt{CLK\_I} of the Wishbone specification~\cite{wishbone}.

\item[\texttt{async\_rst\_i}] Optional asynchronous reset input for all sequential logic. \\
  This reset signal may assert asynchronously, but must deassert synchronously. If no
  asynchrounous reset is implemented, this input must be tied to zero.

\item[\texttt{sync\_rst\_i}] Synchronous reset input. \\
  For WbXBC components, this synchronous reset is not required, if an asynchronous reset is provided.
  If no synchrounous reset is implemented, this input must be tied to zero.
  This reset input corresponds to signal \texttt{RST\_I} of the Wishbone specification~\cite{wishbone}.
 
\end{description}


\subsubsection{Program Bus}
\label{integration:if:pbus}

\begin{description}[style=nextline]
  
\item[\texttt{pbus\_cyc\_o}] Cycle indicator output. \\
  This output signal corresponds to signal \texttt{CYC\_O} of the Wishbone specification~\cite{wishbone}.

\item[\texttt{pbus\_stb\_o}] Strobe output. \\   
  This output signal corresponds to signal \texttt{STB\_O} of the Wishbone specification~\cite{wishbone}.

\item[\texttt{pbus\_we\_o}]  Write enable output. \\
  This output signal corresponds to signal \texttt{WE\_O} of the Wishbone specification~\cite{wishbone}.

\item[\texttt{pbus\_adr\_o}] Address bus. \\   
  These output signals correspond to bus \texttt{ADR\_O} of the Wishbone specification~\cite{wishbone}.

\item[\texttt{pbus\_dat\_o}] Write data bus. \\    
  These output signals correspond to bus \texttt{DAT\_O} of the Wishbone specification~\cite{wishbone}.

\item[\texttt{pbus\_tga\_o}] Address bus tags. \\   
  These output signals correspond to bus \texttt{TGA\_O} of the Wishbone specification~\cite{wishbone}.

\item[\texttt{pbus\_ack\_i}] Acknowlede input. \\   
  This input signal corresponds to signal \texttt{ACK\_I} of the Wishbone specification~\cite{wishbone}.

\item[\texttt{pbus\_err\_i}] Error indicator input. \\  
  This input signal corresponds to signal \texttt{ERR\_I} of the Wishbone specification~\cite{wishbone}.

\item[\texttt{pbus\_stall\_i}] Pipeline stall input. \\
  This input signal corresponds to signal \texttt{STALL\_I} of the Wishbone specification~\cite{wishbone}.

\item[\texttt{pbus\_dat\_i}] Read data bus. \\ 
  These input signals correspond to bus \texttt{DAT\_I} of the Wishbone specification~\cite{wishbone}.

\end{description}




\subsubsection{Stack Bus}
\label{integration:if:sbus}

\begin{description}[style=nextline]

\item[\texttt{sbus\_cyc\_o}] Cycle indicator output. \\
  This output signal corresponds to signal \texttt{CYC\_O} of the Wishbone specification~\cite{wishbone}.

\item[\texttt{sbus\_stb\_o}] Strobe output. \\   
  This output signal corresponds to signal \texttt{STB\_O} of the Wishbone specification~\cite{wishbone}.

\item[\texttt{sbus\_we\_o}]  Write enable output. \\
  This output signal corresponds to signal \texttt{WE\_O} of the Wishbone specification~\cite{wishbone}.

\item[\texttt{sbus\_adr\_o}] Address bus. \\   
  These output signals correspond to bus \texttt{ADR\_O} of the Wishbone specification~\cite{wishbone}.

\item[\texttt{sbus\_dat\_o}] Write data bus. \\    
  These output signals correspond to bus \texttt{DAT\_O} of the Wishbone specification~\cite{wishbone}.

\item[\texttt{sbus\_tga\_o}] Address bus tags. \\   
  These output signals correspond to bus \texttt{TGA\_O} of the Wishbone specification~\cite{wishbone}.

\item[\texttt{sbus\_ack\_i}] Acknowlede input. \\   
  This input signal corresponds to signal \texttt{ACK\_I} of the Wishbone specification~\cite{wishbone}.

\item[\texttt{sbus\_stall\_i}] Pipeline stall input. \\
  This input signal corresponds to signal \texttt{STALL\_I} of the Wishbone specification~\cite{wishbone}.

\item[\texttt{sbus\_dat\_i}] Read data bus. \\ 
  These input signals correspond to bus \texttt{DAT\_I} of the Wishbone specification~\cite{wishbone}.

\end{description}

\subsubsection{Interrupt Interface}
\label{integration:if:irq}

\begin{description}[style=nextline]

\item[\texttt{irq\_ack\_o}] Interrupt acknowledge. \\
  TBD

\item[\texttt{irq\_req\_i}] Interrupt request. \\
  TBD

\end{description}

\subsubsection{Probe Signals}
\label{integration:if:prb}








\subsection{Target Specific Design Files}
\label{integration:ifs}




