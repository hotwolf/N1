%###############################################################################
%# N1 - Manual - Overview                                                      #
%###############################################################################
%#    Copyright 2018 Dirk Heisswolf                                            #
%#    This file is part of the N1 project.                                     #
%#                                                                             #
%#    N1 is free software: you can redistribute it and/or modify               #
%#    it under the terms of the GNU General Public License as published by     #
%#    the Free Software Foundation, either version 3 of the License, or        #
%#    (at your option) any later version.                                      #
%#                                                                             #
%#    N1 is distributed in the hope that it will be useful,                    #
%#    but WITHOUT ANY WARRANTY; without even the implied warranty of           #
%#    MERCHANTABILITY or FITNESS FOR A PARTICULAR PURPOSE.  See the            #
%#    GNU General Public License for more details.                             #
%#                                                                             #
%#    You should have received a copy of the GNU General Public License        #
%#    along with N1.  If not, see <http://www.gnu.org/licenses/>.              #
%###############################################################################
%# Version History:                                                            #
%#   Novemmber 26, 2018                                                        #
%#      - Initial release                                                      #
%###############################################################################

\section{Overview}
\label{overview}

The N1 is a 16-bit stack machine, targeted for low-end FPGA applications.
Its instruction set and architecture are designed for efficient execution
of \Gls{forth} code. 

\noindent
Here is a summary of the N1's characteristics:

\begin{description}[style=nextline]

%Memory connection  
\item[Memory connection:]  
  \begin{itemize}
  \item[]
  \item 16-bit \gls{vna}
  \item Separate address space for \gls{stack} content
  \item \Gls{wb} interfaces to main and stack memory 
  \item Up to 128KB (main) memory space
  \item Memory addressable in 16-bit entities only
  \end{itemize}

%Stacks  
\item[Stacks:]
  \begin{itemize}
  \item[]
  \item Two hardware \glspl{stack} (\glslink{ps}{parameter} and \gls{rs})
  \item Each \gls{stack} consists of three segments:
    \begin{description}[style=nextline]
    %Upper stack  
    \item[\Gls{us}:]
      \begin{itemize}
      \item[]
      \item Shift registers with selectable shift direction for each individual cell
      \item Fixed size
        \begin{itemize}
        \item Upper \gls{ps}: 4 \glspl{cell}
        \item Upper \gls{rs}: 1 \gls{cell}
        \end{itemize}
      \end{itemize}
    %Intermediate stack  
    \item[\Gls{is}:]
      \begin{itemize}
      \item[]
      \item Buffer with lazy data transfers to and from the lower stack
      \item Configurable size
      \end{itemize}
    %Lower stack
    \item[\Gls{ls}:]
      \begin{itemize}
      \item[]
      \item \gls{ram} space shared by both \glspl{stack}
      \item \Glspl{stack} grow towards each other
      \item Up to 128KB in size
      \end{itemize}
    \end{description}
  \end{itemize}

%Instruction set  
\item[Instruction set:]
  \begin{itemize}
  \item[]
  \item Fixed instruction size of 16-bit
  \item \Glspl{jump} and \glspl{call}
    \begin{itemize}
    \item \Gls{indadr}
    \item \Gls{diradr} within a 32KB window
    \item Two bus cycle execution time
    \item Return from \glspl{call} performed concurrently with last instruction
    \end{itemize}
  \item \Glspl{branch}
    \begin{itemize}
    \item \glslink{diradr}{Direct} \gls{reladr} within a 16KB range
    \item Two bus cycles of execution time if branch is taken, one cycle if not
    \end{itemize}         
  \item \Glspl{literal}
    \begin{itemize}
    \item \glslink{immop}{Immediate} encoding of literals between -2048 and 2047
    \item Literals out of this range requre one additional instruction
    \end{itemize}
  \item \glslink{alu}{Arithmetic and logic operations}
    \begin{itemize}
    \item Single cycle \Gls{alu} operations include:
      \begin{itemize}
      \item Sum and Difference
      \item Comparisons
      \item Signed and unsigned products
      \item Bitwise logic operations
      \item Milti-bit shifts
     \end{itemize}
    \item Optional \glslink{immop}{immediate} encoding of one operand, using
      5-bit encoding 
    \end{itemize}
  \item \Gls{stack} operations
    \begin{itemize}
      \item All 1024 stack transitions of the \gls{us} encodable
    \end{itemize}
  \item Memory I/O
    \begin{itemize}
    \item \Gls{indadr}
    \item \Gls{diradr} within a 511B window
    \item Two bus cycle execution time if branch is taken, one cycle if not
    \end{itemize}   
  \end{itemize}

%Exceptions
\item[Exceptions:]
  \begin{itemize}
  \item[]
  \item Exception handler invoked by five error conditions:
    \begin{itemize}
    \item \Gls{ps} overflow
    \item \Gls{ps} underflow
    \item \Gls{rs} overflow
    \item \Gls{rs} underflow
    \item Access violations in the (main) address space
    \end{itemize}
  \end{itemize}
    
%Interrupts  
\item[Interrupts:]
  \begin{itemize}
  \item[]
  \item Optional interrupt support through external interrupt controller
  \item Automatic interrupt acknowledge (flag clearing) supported
  \end{itemize}
  
\end{description}
