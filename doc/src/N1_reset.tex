%###############################################################################
%# N1 - Manual - Reset, Exceptions, and Interrupts                             #
%###############################################################################
%#    Copyright 2018 - 2019 Dirk Heisswolf                                     #
%#    This file is part of the N1 project.                                     #
%#                                                                             #
%#    N1 is free software: you can redistribute it and/or modify               #
%#    it under the terms of the GNU General Public License as published by     #
%#    the Free Software Foundation, either version 3 of the License, or        #
%#    (at your option) any later version.                                      #
%#                                                                             #
%#    N1 is distributed in the hope that it will be useful,                    #
%#    but WITHOUT ANY WARRANTY; without even the implied warranty of           #
%#    MERCHANTABILITY or FITNESS FOR A PARTICULAR PURPOSE.  See the            #
%#    GNU General Public License for more details.                             #
%#                                                                             #
%#    You should have received a copy of the GNU General Public License        #
%#    along with N1.  If not, see <http://www.gnu.org/licenses/>.              #
%###############################################################################
%# Version History:                                                            #
%#   March 4, 2019                                                             #
%#      - Initial release                                                      #
%###############################################################################

\section{Reset, Exceptions, and Interrupts}
\label{reset}

There are three hardware mechanisms in the N1 processor, which can stop the ongoing
program flow in order to react to an urgent hardware condition:
Reset, Exceptions and Interrupts.

\subsection{Reset}
\label{reset:rst}
A reset puts the entire sequential logic of the N1 into a defined initial state.
The \gls{rs} becomes completely cleared and the \gls{ps} is initialized
to hold exactly one cell, containing the reset indicator \texttt{0x0000} (see \tabref{reset:tc}).
After every reset, program execution will begin at address \texttt{0x0000}.
Any context of the previous program flow is lost.
Resets are generated by the system's hardware and occur at least once during power-up.

\subsection{Exceptions}
\label{reset:excpt}
Exceptions are triggered by error conditions and allow the software to restore the functionality
of the system. There are five error conditions, which can be detected by the N1 hardware:
\begin{description}[style=nextline]
\item[\Gls{ps} overflow]
  A \gls{ps} overflow occurs when the capacity of the lower stack's RAM is exceeded
  (excluding a little margin, which is required for the error handling). 
\item[\Gls{rs} stack underflow]
  A \gls{ps} underflow occurs when an instruction requires more arguments than
  available on the \gls{stack} and when a stack instruction would result in non-continuous filling
  of the stack.
\item[\Gls{rs} overflow]
  A \gls{rs} overflow occurs when the capacity of the lower stack's RAM is exceeded
  (excluding a little margin, which is required for the error handling). 
\item[\Gls{rs} underflow]
  A \gls{rs} underflow occurs when an instruction requires more arguments than
  available on the \gls{rs}.
\item[Address out of range]
  This error condition indicates a memory access to a restricted address. This can either
  be caused by an instruction fetch or a data access
\end{description}
In any of these cases, the N1 processor will push a \gls{tc} (see \tabref{reset:tc}) onto
the \gls{ps} and proceed with code execution at address \texttt{0x0000}.
The \gls{rs} and the lower content of the \gls{ps} remain untouched.
The context of the previous program execution is not reserved.
To avoid reoccurance of error conditions during the execution of the handler routine, excepions are
temorarily disabled after detection. Exceptions must then be reenabled by a control instruction
(see \tabref{opcodes:ctrl:smpl})  when the error is resolved.
The \glspl{tc} listed in \tabref{reset:tc} comply with the exception word set of
the ANS Forth standard~\cite{dpans94}.

\begingroup
\setlength{\LTleft}{-20cm plus -1fill}
\setlength{\LTright}{\LTleft}
\begin{center}
  \rowcolors{1}{gray!12}{white}                                         %set alternating row color
  \begin{longtable}{|c|l|}
    \rowcolor{white}
    \caption{Throw codes}
    \label{reset:tc} \\
    %Header
    \hline                                     
    \rowcolor{gray!25}
    \multicolumn{1}{|c|}{\textbf{\rule{0pt}{2.5ex}Throw Code}}     &  
    \multicolumn{1}{c|}{\textbf{\rule{0pt}{2.5ex}Condition}}\\
    \hline
    \endhead                               
    %Footers
    \hline
    \rowcolor{white}
    \multicolumn{2}{r}{\tiny{...continued}} \\
    \endfoot
    \hline
    \endlastfoot

    %Reset
    \texttt{0x0000} (0)                 &    
    Reset                               \\* \hline

    %Parameter stack overflow
    \texttt{0xFFFD} (-3)                &    
    Parameter stack overflow            \\* \hline

    %Parameter stack underflow
    \texttt{0xFFFC} (-4)                &    
      Parameter stack underflow         \\* \hline

    % stack overflow
    \texttt{0xFFFB} (-5)                &    
    Parameter stack overflow            \\* \hline

    %Parameter stack underflow
    \texttt{0xFFFA} (-6)                &    
    Parameter stack underflow           \\* \hline

    %Invalid memory address
    \texttt{0xFFF7} (-9)                &    
    Invalid memory address              \\* \hline

  \end{longtable}
\end{center}  
\endgroup

\noindent
The five hardware exceptions can be easily complemented by user defined software exceptions.
Software exceptions can be thrown by pushing a unique \gls{tc} onto the \gls{ps} and performing
a \gls{jump} to address \texttt{0x0000}.
Hardware and software exceptions can then be handled by a common exception handler routine.

\subsection{Interrupts}
\label{reset:irq}
Interrupts are service requests which are generated by the peripheral hardware. They cause
a temporary interruption of the ongoing program flow. 
When an iterrupt occurs, the program counter is saved to the \gls{rs} and an interrupt service
routine is executed. The location of the interrupt service routine is determined by the system's
interrupt controller hardware. Further interrupts are automatically disabled during the execution
of the interrupt service routine and must be manually reenabled by a control instruction
(see \tabref{opcodes:ctrl:smpl}) before resuming the interrupted program flow.
