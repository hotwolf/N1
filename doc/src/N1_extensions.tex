%###############################################################################
%# N1 - Manual - Instruction Set Extensions                                    #
%###############################################################################
%#    Copyright 2018 - 2024 Dirk Heisswolf                                     #
%#    This file is part of the N1 project.                                     #
%#                                                                             #
%#    N1 is free software: you can redistribute it and/or modify               #
%#    it under the terms of the GNU General Public License as published by     #
%#    the Free Software Foundation, either version 3 of the License, or        #
%#    (at your option) any later version.                                      #
%#                                                                             #
%#    N1 is distributed in the hope that it will be useful,                    #
%#    but WITHOUT ANY WARRANTY; without even the implied warranty of           #
%#    MERCHANTABILITY or FITNESS FOR A PARTICULAR PURPOSE.  See the            #
%#    GNU General Public License for more details.                             #
%#                                                                             #
%#    You should have received a copy of the GNU General Public License        #
%#    along with N1.  If not, see <http://www.gnu.org/licenses/>.              #
%###############################################################################
%# Version History:                                                            #
%#   October 11, 2019                                                          #
%#      - Initial release                                                      #
%#   February 13, 2023                                                         #
%#      - Added interrupt and KEY/EMIT extensions                              #
%###############################################################################

\section{Extensions}
\label{extensions}

The instruction set of the N1 processor (see \secref{opcodes}) reserves a number
of undefined \glspl{opcode} for functional extensions.
These extensions imply a trade-off between hardware complexity and functional 
improvements.
They can be selected individually for each system integrating the N1 processor
(see \secref{integration}).    

%ROT extension
\subsection{ROT Extension}
\label{extensions:rot}

The \gls{rotext} adds two data paths to the \gls{us}, allowing direct data transfers
between the top and the third element of the \gls{ps}. These new stack transitions 
are performed by the regular stack instructions (see \secref{opcodes:stack}), using
some of the reserved stack transition patterns. 
\figref{extensions:rot:transpat} illustrates the usage of the \gls{rotext}. 

\begin{figure}[!h]
  %\begin{center}
  \makebox[\textwidth][c]{
    \scalebox{0.72} {
      \begin{tikzpicture}
        
        %Upwards
        \begin{scope}[shift={(0,14)}]
          
          %Stack instruction
          \draw [thick, fill=gray!3]  (1,4) rectangle (17,5);
          \draw [thick]               (1,4) rectangle  (2,5);
          \draw [thick]               (2,4) rectangle  (7,5); 
          \draw [thick, fill=white]   (7,4) rectangle  (8,5); 
          \draw [thick, fill=white]   (8,4) rectangle (10,5); 
          \draw [thick, fill=white]   (10,4) rectangle (12,5); 
          \draw [thick, fill=white]   (12,4) rectangle (13,5); 
          \draw [thick, fill=gray!48] (13,4) rectangle (14,5); 
          \draw [thick, fill=white]   (14,4) rectangle (15,5); 
          \draw [thick, fill=gray!48] (14,4) rectangle (15,5); 
          \draw [thick, fill=white]   (16,4) rectangle (17,5); 
          
          \node [above] at  (1.5,5) {15};
          \node [above] at  (2.5,5) {14};
          \node [above] at  (3.5,5) {13};
          \node [above] at  (4.5,5) {12};
          \node [above] at  (5.5,5) {11};
          \node [above] at  (6.5,5) {10};
          \node [above] at  (7.5,5) {9};
          \node [above] at  (8.5,5) {8};
          \node [above] at  (9.5,5) {7};
          \node [above] at (10.5,5) {6};
          \node [above] at (11.5,5) {5};
          \node [above] at (12.5,5) {4};
          \node [above] at (13.5,5) {3};
          \node [above] at (14.5,5) {2};
          \node [above] at (15.5,5) {1};
          \node [above] at (16.5,5) {0};
          
          \node         at (1.5,4.5)     {\huge{\texttt{;}}};
          \node         at (2.5,4.5)     {\huge{\texttt{0}}};
          \node         at (3.5,4.5)     {\huge{\texttt{0}}};
          \node         at (4.5,4.5)     {\huge{\texttt{0}}};
          \node         at (5.5,4.5)     {\huge{\texttt{0}}};   
          \node         at (6.5,4.5)     {\huge{\texttt{1}}};   
          
          \node         at (7.5,4.5)     {\Large{IST}};   
          \node         at (9,4.5)       {\Large{UST}};   
          \node         at (11,4.5)      {\Large{UST}};   
          \node         at (12.48,4.48)  {\small{UST}};
          \node         at (13.5,4.5)    {\huge{\texttt{\textbf{1}}}};             
          \node         at (15.48,4.48)  {\small{UST}};   
          \node         at (14.5,4.5)    {\huge{\texttt{\textbf{1}}}};             
          \node         at (16.5,4.5)    {\Large{IST}};   
          
          \node [below right] at (1,3.8) {\large{Stack Instruction}};
          
          %Bit field association
          \draw [ultra thick, dashed, ->]  (7.5,4)  -- (7.5,3)    -- (2.5,3)    -- (2.5,2);
          \draw [ultra thick, dashed, ->]  (9,4)    -- (9,2.8)    -- (5.5,2.8)  -- (5.5,2);
          \draw [ultra thick, dashed, ->]  (11,4)   -- (11,2.6)   -- (8.5,2.6)  -- (8.5,2);
          \draw [ultra thick, dashed, ->]  (12.5,4) -- (12.5,2.4) -- (11.5,2.4) -- (11.5,2);
          \draw [ultra thick, dashed, ->]  (15.5,4) -- (15.5,2.4) -- (15,2.4)   -- (15,2);
          \draw [ultra thick, dashed, ->]  (16.5,4) -- (16.5,2.4) -- (18,2.4)   -- (18,2);
          
          %Lower parameter stack
          \draw [thick, fill=gray!3]  (1,1) -- (2,1) -- (2,2) -- (1,2);
          \draw [ultra thick, ->]     (2,1.8) --  (3,1.8);
          \draw [ultra thick, <-]     (2,1.2) --  (3,1.2);         
          
          %Upper parameter stack
          \draw [thick, fill=gray!3]  (3,1) rectangle  (5,2);
          \draw [ultra thick, ->]     (5,1.8) --  (6,1.8);
          \draw [ultra thick, <->]    (5,1.5) --  (6,1.5);
          \draw [ultra thick, <-]     (5,1.2) --  (6,1.2);         
          
          \draw [thick, fill=gray!3]  (6,1) rectangle  (8,2);
          \draw [ultra thick, ->]     (8,1.8) --  (9,1.8);
          \draw [ultra thick, <->]    (8,1.5) --  (9,1.5);
          \draw [ultra thick, <-]     (8,1.2) --  (9,1.2);         
          
          \draw [thick, fill=gray!3]  (9,1) rectangle  (11,2);
          %\draw [ultra thick, ->]     (11,1.8) --  (12,1.8);
          %\draw [ultra thick, <->]    (11,1.5) --  (12,1.5);
          \draw [ultra thick, <-]     (11,1.2) --  (12,1.2);         

          \draw [thick, fill=gray!3]  (12,1) rectangle  (14,2);
          \node at (13,1.45)          {\Large{\textbf{TOS}}};        
          \node at (3.5,0.5)          {\large{Parameter stack}};

          %ROT extension
          \draw [rounded corners, ultra thick, ->] (7,1) -- (7,0.5) -- (13,0.5) -- (13,1);

          %Stack boundary
          \draw [ultra thick, ->]     (14,1.8)    --  (15.5,1.8);
          %\draw [ultra thick, <->]    (14,1.5)    --  (15.5,1.5);
          %\draw [ultra thick, <-]     (14,1.2)    --  (15.5,1.2);         
          \draw [dotted]              (14.75,2.2) --  (14.75,0.2);
          
          %Upper return stack
          \draw [thick, fill=gray!24] (15.5,1) rectangle  (17.5,2);
          \node at (16.5,1.45)        {\Large{\textbf{TOS}}};
          \draw [ultra thick, ->]     (17.5,1.8) --  (18.5,1.8);
          \draw [ultra thick, <-]     (17.5,1.2) --  (18.5,1.2);         
          \node at (17.5,0.5)         {\large{Return stack}};
          
          %Lower return stack
          \draw [thick, fill=gray!24] (19.5,1) -- (18.5,1) -- (18.5,2) -- (19.5,2);
          
        \end{scope}
       
        %Downwards        
        \begin{scope}[shift={(0,7)}]
          
          %Stack instruction
          \draw [thick, fill=gray!3]   (1,4) rectangle (17,5);
          \draw [thick]                (1,4) rectangle  (2,5);
          \draw [thick]                (2,4) rectangle  (7,5); 
          \draw [thick, fill=white]    (7,4) rectangle  (8,5); 
          \draw [thick, fill=white]    (8,4) rectangle  (9,5); 
          \draw [thick, fill=gray!48]  (9,4) rectangle (10,5); 
          \draw [thick, fill=gray!48] (10,4) rectangle (11,5); 
          \draw [thick, fill=white]   (11,4) rectangle (12,5); 
          \draw [thick, fill=white]   (12,4) rectangle (14,5); 
          \draw [thick, fill=white]   (14,4) rectangle (16,5); 
          \draw [thick, fill=white]   (16,4) rectangle (17,5); 
          
          \node [above] at  (1.5,5) {15};
          \node [above] at  (2.5,5) {14};
          \node [above] at  (3.5,5) {13};
          \node [above] at  (4.5,5) {12};
          \node [above] at  (5.5,5) {11};
          \node [above] at  (6.5,5) {10};
          \node [above] at  (7.5,5) {9};
          \node [above] at  (8.5,5) {8};
          \node [above] at  (9.5,5) {7};
          \node [above] at (10.5,5) {6};
          \node [above] at (11.5,5) {5};
          \node [above] at (12.5,5) {4};
          \node [above] at (13.5,5) {3};
          \node [above] at (14.5,5) {2};
          \node [above] at (15.5,5) {1};
          \node [above] at (16.5,5) {0};
          
          \node         at (1.5,4.5)     {\huge{\texttt{;}}};
          \node         at (2.5,4.5)     {\huge{\texttt{0}}};
          \node         at (3.5,4.5)     {\huge{\texttt{0}}};
          \node         at (4.5,4.5)     {\huge{\texttt{0}}};
          \node         at (5.5,4.5)     {\huge{\texttt{0}}};   
          \node         at (6.5,4.5)     {\huge{\texttt{1}}};   
          
          \node         at (7.5,4.5)     {\Large{IST}};   
          \node         at (8.48,4.48)   {\small{UST}};
          \node         at (9.5,4.5)     {\huge{\texttt{\textbf{1}}}};             
          \node         at (11.48,4.48)  {\small{UST}};
          \node         at (10.5,4.5)    {\huge{\texttt{\textbf{1}}}};             
          \node         at (13,4.5)      {\Large{UST}};   
          \node         at (15,4.5)      {\Large{UST}};   
          \node         at (16.5,4.5)    {\Large{IST}};   
          
          \node [below right] at (1,3.8) {\large{Stack Instruction}};
          
          %Bit field association
          \draw [ultra thick, dashed, ->]  (7.5,4)  -- (7.5,3)    -- (2.5,3)    -- (2.5,2);
          \draw [ultra thick, dashed, ->]  (8.5,4)  -- (8.5,2.8)  -- (5.5,2.8)  -- (5.5,2);
          \draw [ultra thick, dashed, ->]  (11.5,4) -- (11.5,2.6) -- (8.5,2.6)  -- (8.5,2);
          \draw [ultra thick, dashed, ->]  (13,4)   -- (13,2.4)   -- (11.5,2.4) -- (11.5,2);
          \draw [ultra thick, dashed, ->]  (15,4)   -- (15,2);
          \draw [ultra thick, dashed, ->]  (16.5,4) -- (16.5,2.4) -- (18,2.4) -- (18,2);
          
          %Lower parameter stack
          \draw [thick, fill=gray!3]  (1,1) -- (2,1) -- (2,2) -- (1,2);
          \draw [ultra thick, ->]     (2,1.8) --  (3,1.8);
          \draw [ultra thick, <-]     (2,1.2) --  (3,1.2);         
          
          %Upper parameter stack
          \draw [thick, fill=gray!3]  (3,1) rectangle  (5,2);
          %\draw [ultra thick, ->]     (5,1.8) --  (6,1.8);
          %\draw [ultra thick, <->]    (5,1.5) --  (6,1.5);
          \draw [ultra thick, <-]     (5,1.2) --  (6,1.2);         
          
          \draw [thick, fill=gray!3]  (6,1) rectangle  (8,2);
          \draw [ultra thick, ->]     (8,1.8) --  (9,1.8);
          %\draw [ultra thick, <->]    (8,1.5) --  (9,1.5);
          %\draw [ultra thick, <-]     (8,1.2) --  (9,1.2);         
          
          \draw [thick, fill=gray!3]  (9,1) rectangle  (11,2);
          \draw [ultra thick, ->]     (11,1.8) --  (12,1.8);
          \draw [ultra thick, <->]    (11,1.5) --  (12,1.5);
          \draw [ultra thick, <-]     (11,1.2) --  (12,1.2);         
          
          \draw [thick, fill=gray!3]  (12,1) rectangle  (14,2);
          \node at (13,1.45)          {\Large{\textbf{TOS}}};        
          \node at (3.5,0.5)           {\large{Parameter stack}};

          %ROT extension
          \draw [rounded corners, ultra thick, <-] (7,1) -- (7,0.5) -- (13,0.5) -- (13,1);
         
          %Stack boundary
          \draw [ultra thick, ->]     (14,1.8)    --  (15.5,1.8);
          \draw [ultra thick, <->]    (14,1.5)    --  (15.5,1.5);
          \draw [ultra thick, <-]     (14,1.2)    --  (15.5,1.2);         
          \draw [dotted]              (14.75,2.2) --  (14.75,0.2);
          
          %Upper return stack
          \draw [thick, fill=gray!24] (15.5,1) rectangle  (17.5,2);
          \node at (16.5,1.45)        {\Large{\textbf{TOS}}};
          \draw [ultra thick, ->]     (17.5,1.8) --  (18.5,1.8);
          \draw [ultra thick, <-]     (17.5,1.2) --  (18.5,1.2);         
          \node at (17.5,0.5)         {\large{Return stack}};
          
          %Lower return stack
          \draw [thick, fill=gray!24] (19.5,1) -- (18.5,1) -- (18.5,2) -- (19.5,2);

        \end{scope}
        
        %Exchange        
        \begin{scope}
          
          %Stack instruction
          \draw [thick, fill=gray!3]  (1,4) rectangle (17,5);
          \draw [thick]               (1,4) rectangle  (2,5);
          \draw [thick]               (2,4) rectangle  (7,5); 
          \draw [thick, fill=white]   (7,4) rectangle  (8,5); 
          \draw [thick, fill=white]   (8,4) rectangle  (9,5); 
          \draw [thick, fill=gray!48]  (9,4) rectangle (10,5); 
          \draw [thick, fill=gray!48] (10,4) rectangle (11,5); 
          \draw [thick, fill=white]   (11,4) rectangle (12,5); 
          \draw [thick, fill=white]   (12,4) rectangle (13,5); 
          \draw [thick, fill=gray!48] (13,4) rectangle (14,5); 
          \draw [thick, fill=white]   (14,4) rectangle (15,5); 
          \draw [thick, fill=gray!48] (14,4) rectangle (15,5); 
          \draw [thick, fill=white]   (16,4) rectangle (17,5); 
        
          \node [above] at  (1.5,5) {15};
          \node [above] at  (2.5,5) {14};
          \node [above] at  (3.5,5) {13};
          \node [above] at  (4.5,5) {12};
          \node [above] at  (5.5,5) {11};
          \node [above] at  (6.5,5) {10};
          \node [above] at  (7.5,5) {9};
          \node [above] at  (8.5,5) {8};
          \node [above] at  (9.5,5) {7};
          \node [above] at (10.5,5) {6};
          \node [above] at (11.5,5) {5};
          \node [above] at (12.5,5) {4};
          \node [above] at (13.5,5) {3};
          \node [above] at (14.5,5) {2};
          \node [above] at (15.5,5) {1};
          \node [above] at (16.5,5) {0};
          
          \node         at (1.5,4.5)     {\huge{\texttt{;}}};
          \node         at (2.5,4.5)     {\huge{\texttt{0}}};
          \node         at (3.5,4.5)     {\huge{\texttt{0}}};
          \node         at (4.5,4.5)     {\huge{\texttt{0}}};
          \node         at (5.5,4.5)     {\huge{\texttt{0}}};   
          \node         at (6.5,4.5)     {\huge{\texttt{1}}};   
          
          \node         at (7.5,4.5)     {\Large{IST}};   
          \node         at (8.48,4.48)   {\small{UST}};
          \node         at (9.5,4.5)     {\huge{\texttt{\textbf{1}}}};             
          \node         at (11.48,4.48)  {\small{UST}};
          \node         at (10.5,4.5)    {\huge{\texttt{\textbf{1}}}};             

          \node         at (12.48,4.48)  {\small{UST}};
          \node         at (13.5,4.5)    {\huge{\texttt{\textbf{1}}}};             
          \node         at (15.48,4.48)  {\small{UST}};   
          \node         at (14.5,4.5)    {\huge{\texttt{\textbf{1}}}};             
          \node         at (16.5,4.5)    {\Large{IST}};   

          %\node         at (13,4.5)      {\Large{UST}};   
          %\node         at (15,4.5)      {\Large{UST}};   
          %\node         at (16.5,4.5)    {\Large{IST}};   
          
          \node [below right] at (1,3.8) {\large{Stack Instruction}};
          
          %Bit field association
          \draw [ultra thick, dashed, ->]  (7.5,4)  -- (7.5,3)    -- (2.5,3)    -- (2.5,2);
          \draw [ultra thick, dashed, ->]  (8.5,4)  -- (8.5,2.8)  -- (5.5,2.8)  -- (5.5,2);
          \draw [ultra thick, dashed, ->]  (11.5,4) -- (11.5,2.6) -- (8.5,2.6)  -- (8.5,2);
          \draw [ultra thick, dashed, ->]  (12.5,4) -- (12.5,2.4) -- (11.5,2.4) -- (11.5,2);
          \draw [ultra thick, dashed, ->]  (15.5,4) -- (15.5,2.4) -- (15,2.4)   -- (15,2);
          \draw [ultra thick, dashed, ->]  (16.5,4) -- (16.5,2.4) -- (18,2.4) -- (18,2);
          
          %Lower parameter stack
          \draw [thick, fill=gray!3]  (1,1) -- (2,1) -- (2,2) -- (1,2);
          \draw [ultra thick, ->]     (2,1.8) --  (3,1.8);
          \draw [ultra thick, <-]     (2,1.2) --  (3,1.2);         
          
          %Upper parameter stack
          \draw [thick, fill=gray!3]  (3,1) rectangle  (5,2);
          %\draw [ultra thick, ->]     (5,1.8) --  (6,1.8);
          %\draw [ultra thick, <->]    (5,1.5) --  (6,1.5);
          \draw [ultra thick, <-]     (5,1.2) --  (6,1.2);         
          
          \draw [thick, fill=gray!3]  (6,1) rectangle  (8,2);
          \draw [ultra thick, ->]     (8,1.8) --  (9,1.8);
          %\draw [ultra thick, <->]    (8,1.5) --  (9,1.5);
          %\draw [ultra thick, <-]     (8,1.2) --  (9,1.2);         
          
          \draw [thick, fill=gray!3]  (9,1) rectangle  (11,2);
          %\draw [ultra thick, ->]     (11,1.8) --  (12,1.8);
          %\draw [ultra thick, <->]    (11,1.5) --  (12,1.5);
          \draw [ultra thick, <-]     (11,1.2) --  (12,1.2);         
          
          \draw [thick, fill=gray!3]  (12,1) rectangle  (14,2);
          \node at (13,1.45)          {\Large{\textbf{TOS}}};        
          \node at (3.5,0.5)           {\large{Parameter stack}};

          %ROT extension
          \draw [rounded corners, ultra thick, <->] (7,1) -- (7,0.5) -- (13,0.5) -- (13,1);
         
          %Stack boundary
          \draw [ultra thick, ->]     (14,1.8)    --  (15.5,1.8);
          %\draw [ultra thick, <->]    (14,1.5)    --  (15.5,1.5);
          %\draw [ultra thick, <-]     (14,1.2)    --  (15.5,1.2);         
          \draw [dotted]              (14.75,2.2) --  (14.75,0.2);
          
          %Upper return stack
          \draw [thick, fill=gray!24] (15.5,1) rectangle  (17.5,2);
          \node at (16.5,1.45)        {\Large{\textbf{TOS}}};
          \draw [ultra thick, ->]     (17.5,1.8) --  (18.5,1.8);
          \draw [ultra thick, <-]     (17.5,1.2) --  (18.5,1.2);         
          \node at (17.5,0.5)         {\large{Return stack}};
          
          %Lower return stack
          \draw [thick, fill=gray!24] (19.5,1) -- (18.5,1) -- (18.5,2) -- (19.5,2);

        \end{scope}
        
      \end{tikzpicture}
    }
  }
  \caption{Stack transitions of the \gls{rotext}}
  \label{extensions:rot:transpat}
  %\end{center}
\end{figure}

\subsubsection{Accellerated Stack Operations}
\label{extensions:rot:ops}

The \gls{rotext} improves the execution time and code density of the three commmon
stack operations \texttt{TUCK}, \texttt{ROT}, and \texttt{-ROT}
(see \tabref{extensions:rot:mapping}). 
This means that all common single-cell \gls{ps} operations shown in
\tabref{opcodes:stack:mapping} can be executed in one cycle if the \gls{rotext} is
enabled.

\begingroup
\setlength{\LTleft}{-20cm plus -1fill}
\setlength{\LTright}{\LTleft}
\begin{center}
  \rowcolors{1}{gray!12}{white}                                         %set alternating row color
  \begin{longtable}{|c|c|c|c|}
    \rowcolor{white}
    \caption{Improved stack operations}
    \label{extensions:rot:mapping} \\
    %Header
    \hline                                     
    \rowcolor{gray!25}
    \multicolumn{1}{|c|}{\textbf{\rule{0pt}{2.5ex}Word}}       &  
    \multicolumn{1}{c|}{\textbf{\rule{0pt}{2.5ex}Description}} & 
    \multicolumn{1}{c|}{\textbf{\rule{0pt}{2.5ex}Transitions}} & 
    \multicolumn{1}{c|}{\textbf{\rule{0pt}{2.5ex}Opcode}} \\
    \hline
    \endhead                               
    %Footers
    \hline
    \rowcolor{white}
    \multicolumn{4}{r}{\tiny{...continued}} \\
    \endfoot
    \hline
    \endlastfoot

    %TUCK
    \texttt{TUCK} &
    ( x1 x2 -- x2 x1 x2 ) &
    \multicolumn{1}{m{21.35em}|}{
    \scalebox{0.4} {
      \begin{tikzpicture}
        \begin{scope}[shift={(0,0)}]
          \draw [thick, fill=gray!3]  (0,0.5) -- (1,0.5) -- (1,1.5) -- (0,1.5);%
          \draw [line width=1ex, <-] (1,1) -- (2,1);                           %
          \draw [thick, fill=gray!3]  (2,0.5) rectangle (4,1.5);               %PS+3
          \draw [line width=1ex, <-] (4,1) -- (5,1);                           %
          \draw [thick, fill=gray!3]  (5,0.5) rectangle (7,1.5);               %PS+2
          %\draw [line width=1ex, <->] (7,1) -- (8,1);                         %
          \draw [thick, fill=gray!3]  (8,0.5) rectangle (10,1.5);              %PS+1
          %\draw [line width=1ex, --] (10,1) -- (11,1);                        %
          \draw [thick, fill=gray!3]  (11,0.5) rectangle (13,1.5);             %PS TOS
          \node at (12,0.95)          {\Large{\textbf{TOS}}};                  %
          %\draw [line width=1ex, --] (13,1)  -- (14.5,1);                     %
          \draw [rounded corners, line width=1ex, <-] (6,0.5) -- (6,0) -- (12,0) -- (12,0.5);
          \draw [dotted]              (13.75,0.5) -- (13.75,1.7);              %
          \draw [thick, fill=gray!24] (14.5,0.5) rectangle (16.5,1.5);         %RS TOS
          \node at (15.5,0.95)        {\Large{\textbf{TOS}}};                  %
          %\draw [line width=1ex, --] (16.5,1) -- (17.5,1);                    % 
          \draw [thick, fill=gray!24] (18.5,0.5) -- (17.5,0.5) -- (17.5,1.5) -- (18.5,1.5);
        \end{scope}
       \end{tikzpicture}
    }} &
    \multicolumn{1}{m{4.25em}|}{
    \makecell[c]{ 
      \texttt{0x07C0}
    }} \\ \hline

    %ROT
    \texttt{ROT} &
    ( x1 x2 x3 -- x2 x3 x1 ) &
    \multicolumn{1}{m{21.35em}|}{
    \scalebox{0.4} {
      \begin{tikzpicture}
        \begin{scope}[shift={(0,0)}]
          \draw [thick, fill=gray!3]  (0,0.5) -- (1,0.5) -- (1,1.5) -- (0,1.5);%
          %\draw [line width=1ex, --] (1,1) -- (2,1);                          %
          \draw [thick, fill=gray!3]  (2,0.5) rectangle (4,1.5);               %PS+3
          %\draw [line width=1ex, --] (4,1) -- (5,1);                          %
          \draw [thick, fill=gray!3]  (5,0.5) rectangle (7,1.5);               %PS+2
          \draw [line width=1ex, <-]  (7,1) -- (8,1);                          %
          \draw [thick, fill=gray!3]  (8,0.5) rectangle (10,1.5);              %PS+1
          \draw [line width=1ex, <-]  (10,1) -- (11,1);                        %
          \draw [thick, fill=gray!3]  (11,0.5) rectangle (13,1.5);             %PS TOS
          \node at (12,0.95)          {\Large{\textbf{TOS}}};                  %
          %\draw [line width=1ex, --] (13,1)  -- (14.5,1);                     %
          \draw [rounded corners, line width=1ex, ->] (6,0.5) -- (6,0) -- (12,0) -- (12,0.5);
          \draw [dotted]              (13.75,0.5) -- (13.75,1.7);              %
          \draw [thick, fill=gray!24] (14.5,0.5) rectangle (16.5,1.5);         %RS TOS
          \node at (15.5,0.95)        {\Large{\textbf{TOS}}};                  %
          %\draw [line width=1ex, --] (16.5,1) -- (17.5,1);                    % 
          \draw [thick, fill=gray!24] (18.5,0.5) -- (17.5,0.5) -- (17.5,1.5) -- (18.5,1.5);
        \end{scope}
      \end{tikzpicture}
    }} &
    \multicolumn{1}{m{4.25em}|}{
    \makecell[c]{ 
      \texttt{0x041C}
    }} \\ \hline
    
    %-ROT
    \texttt{-ROT} &
    ( x1 x2 x3 -- x3 x1 x2 ) &
    \multicolumn{1}{m{21.35em}|}{
    \scalebox{0.4} {
      \begin{tikzpicture}
        \begin{scope}[shift={(0,0)}]
          \draw [thick, fill=gray!3]  (0,0.5) -- (1,0.5) -- (1,1.5) -- (0,1.5);%
          %\draw [line width=1ex, --] (1,1) -- (2,1);                          %
          \draw [thick, fill=gray!3]  (2,0.5) rectangle (4,1.5);               %PS+3
          %\draw [line width=1ex, --] (4,1) -- (5,1);                          %
          \draw [thick, fill=gray!3]  (5,0.5) rectangle (7,1.5);               %PS+2
          \draw [line width=1ex, ->]  (7,1) -- (8,1);                          %
          \draw [thick, fill=gray!3]  (8,0.5) rectangle (10,1.5);              %PS+1
          \draw [line width=1ex, ->]  (10,1) -- (11,1);                        %
          \draw [thick, fill=gray!3]  (11,0.5) rectangle (13,1.5);             %PS TOS
          \node at (12,0.95)          {\Large{\textbf{TOS}}};                  %
          %\draw [line width=1ex, --] (13,1)  -- (14.5,1);                     %
          \draw [rounded corners, line width=1ex, <-] (6,0.5) -- (6,0) -- (12,0) -- (12,0.5);
          \draw [dotted]              (13.75,0.5) -- (13.75,1.7);              %
          \draw [thick, fill=gray!24] (14.5,0.5) rectangle (16.5,1.5);         %RS TOS
          \node at (15.5,0.95)        {\Large{\textbf{TOS}}};                  %
          %\draw [line width=1ex, --] (16.5,1) -- (17.5,1);                    % 
          \draw [thick, fill=gray!24] (18.5,0.5) -- (17.5,0.5) -- (17.5,1.5) -- (18.5,1.5);
        \end{scope}
      \end{tikzpicture}
    }} &
    \multicolumn{1}{m{4.25em}|}{
    \makecell[c]{ 
      \texttt{0x04E0}
    }} \\ \hline

  \end{longtable}
\end{center}  
\endgroup

N1 processors with \gls{rotext} are backward compatible to the ones without.
All stack operations can still be executed as listed in \tabref{opcodes:stack:mapping},
even if the \gls{rotext} is enabled.

\subsubsection{Stack Underflow Detection}
\label{opcodes:stack:uf}

The \gls{rotext} introduces three new stack underflow detection rules.
These rules are listed in \tabref{opcodes:stack:ufrules}.

\begingroup
\setlength{\LTleft}{-20cm plus -1fill}
\setlength{\LTright}{\LTleft}
\begin{center}
  \rowcolors{1}{gray!12}{white}                                         %set alternating row color
  \begin{longtable}{|c|c|c|}
    \rowcolor{white}
    \caption{Rules of Stack Underflow Detection}
    \label{opcodes:stack:ufrules} \\
    %Header
    \hline                                     
    \rowcolor{gray!25}
    \multicolumn{1}{|c|}{\textbf{\rule{0pt}{2.5ex}Rule}}       &  
    \multicolumn{1}{c|}{\textbf{\rule{0pt}{2.5ex}Transitions}} & 
    \multicolumn{1}{c|}{\textbf{\rule{0pt}{2.5ex}Description}} \\
    \hline
    \endhead                               
    %Footers
    \hline
    \rowcolor{white}
    \multicolumn{3}{r}{\tiny{...continued}} \\
    \endfoot
    \hline
    \endlastfoot

    %TUCK rule
    \multicolumn{1}{|m{3em}|}{
    \makecell[c]{ 
      ``\texttt{TUCK}''\\
      Rule}} &
    \multicolumn{1}{m{11.5em}|}{
    \scalebox{0.4} {
      \begin{tikzpicture}
        \begin{scope}[shift={(0,0)}]
          \draw [thick, fill=gray!3]  (3,0.5) -- (4,0.5) -- (4,1.5) -- (3,1.5);
          \draw [line width=1ex, <-] (4,1) -- (5,1);                           
          \draw [thick, fill=gray!3]  (5,0.5) rectangle (7,1.5);               
          %\draw [line width=1ex, <->] (7,1) -- (8,1);                         
          \draw [thick, fill=gray]  (8,0.5) rectangle (10,1.5);              
          %\draw [line width=1ex, --] (10,1) -- (11,1);                        
          \draw [thick, fill=gray]  (11,0.5) rectangle (13,1.5);             
          \node[text=gray!3] at (12,0.95)          {\Large{\textbf{TOS}}};                  
          %\draw [line width=1ex, --] (13,1)  -- (14.5,1);                       
          \draw [rounded corners, line width=1ex, <-] (6,0.5) -- (6,0) -- (12,0) -- (12,0.5);
        \end{scope}
      \end{tikzpicture}
    }} &
    %\multicolumn{1}{m{28em}|}{
    \makecell[l]{
    \begin{minipage}[t]{\linewidth}%  
      If the downward \gls{rotext} path is used and the target cell is shifted further downward,
      then the \gls{ps} must hold at least \textbf{two} values prior to the stack operation.
    \end{minipage}%
    } \\ \hline

    %TUCK rule
    \multicolumn{1}{|m{3em}|}{
    \makecell[c]{ 
      ``\texttt{ROT}''\\
      Rule}} &
    \multicolumn{1}{m{11.5em}|}{
    \scalebox{0.4} {
      \begin{tikzpicture}
        \path []  (3,8)   -- (3,9.7);
        
        \begin{scope}[shift={(0,0)}]
          \draw [thick, fill=gray!3]  (3,0.5) -- (4,0.5) -- (4,1.5) -- (3,1.5);
          \draw [line width=1ex, ->] (4,1) -- (5,1);                           
          \draw [thick, fill=gray]  (5,0.5) rectangle (7,1.5);               
          %\draw [line width=1ex, <->] (7,1) -- (8,1);                         
          \draw [thick, fill=gray]  (8,0.5) rectangle (10,1.5);              
          %\draw [line width=1ex, --] (10,1) -- (11,1);                        
          \draw [thick, fill=gray]  (11,0.5) rectangle (13,1.5);             
          \node[text=gray!3] at (12,0.95)          {\Large{\textbf{TOS}}};                  
          %\draw [line width=1ex, --] (13,1)  -- (14.5,1);                       
          \draw [rounded corners, line width=1ex, ->] (6,0.5) -- (6,0) -- (12,0) -- (12,0.5);
        \end{scope}

        \begin{scope}[shift={(0,2)}]
          \draw [thick, fill=gray!3]  (3,0.5) -- (4,0.5) -- (4,1.5) -- (3,1.5);
          %\draw [line width=1ex, <-] (4,1) -- (5,1);                           
          \draw [line width=1ex, -|] (4,1) -- (4.4,1);
          \draw [line width=1ex, |-] (4.6,1) -- (5,1);
          \draw [thick, fill=gray]  (5,0.5) rectangle (7,1.5);               
          %\draw [line width=1ex, <->] (7,1) -- (8,1);                         
          \draw [thick, fill=gray]  (8,0.5) rectangle (10,1.5);              
          %\draw [line width=1ex, --] (10,1) -- (11,1);                        
          \draw [thick, fill=gray]  (11,0.5) rectangle (13,1.5);             
          \node[text=gray!3] at (12,0.95)          {\Large{\textbf{TOS}}};                  
          %\draw [line width=1ex, --] (13,1)  -- (14.5,1);                       
          \draw [rounded corners, line width=1ex, ->] (6,0.5) -- (6,0) -- (12,0) -- (12,0.5);
        \end{scope}

        \begin{scope}[shift={(0,4)}]
          \draw [thick, fill=gray!3]  (3,0.5) -- (4,0.5) -- (4,1.5) -- (3,1.5);
          \draw [line width=1ex, <-] (4,1) -- (5,1);                           
          \draw [thick, fill=gray]  (5,0.5) rectangle (7,1.5);               
          %\draw [line width=1ex, <->] (7,1) -- (8,1);                         
          \draw [thick, fill=gray]  (8,0.5) rectangle (10,1.5);              
          %\draw [line width=1ex, --] (10,1) -- (11,1);                        
          \draw [thick, fill=gray]  (11,0.5) rectangle (13,1.5);             
          \node[text=gray!3] at (12,0.95)          {\Large{\textbf{TOS}}};                  
          %\draw [line width=1ex, --] (13,1)  -- (14.5,1);                       
          \draw [rounded corners, line width=1ex, ->] (6,0.5) -- (6,0) -- (12,0) -- (12,0.5);
        \end{scope}

        \begin{scope}[shift={(0,6)}]
          \draw [thick, fill=gray!3]  (3,0.5) -- (4,0.5) -- (4,1.5) -- (3,1.5);
          \draw [line width=1ex, ->] (4,1) -- (5,1);                           
          \draw [thick, fill=gray]  (5,0.5) rectangle (7,1.5);               
          %\draw [line width=1ex, <->] (7,1) -- (8,1);                         
          \draw [thick, fill=gray]  (8,0.5) rectangle (10,1.5);              
          %\draw [line width=1ex, --] (10,1) -- (11,1);                        
          \draw [thick, fill=gray]  (11,0.5) rectangle (13,1.5);             
          \node[text=gray!3] at (12,0.95)          {\Large{\textbf{TOS}}};                  
          %\draw [line width=1ex, --] (13,1)  -- (14.5,1);                       
          \draw [rounded corners, line width=1ex, <-] (6,0.5) -- (6,0) -- (12,0) -- (12,0.5);
        \end{scope}
        
        \begin{scope}[shift={(0,8)}]
          \draw [thick, fill=gray!3]  (3,0.5) -- (4,0.5) -- (4,1.5) -- (3,1.5);
          %\draw [line width=1ex, <-] (4,1) -- (5,1);                           
          \draw [line width=1ex, -|] (4,1) -- (4.4,1);
          \draw [line width=1ex, |-] (4.6,1) -- (5,1);
          \draw [thick, fill=gray]  (5,0.5) rectangle (7,1.5);               
          %\draw [line width=1ex, <->] (7,1) -- (8,1);                         
          \draw [thick, fill=gray]  (8,0.5) rectangle (10,1.5);              
          %\draw [line width=1ex, --] (10,1) -- (11,1);                        
          \draw [thick, fill=gray]  (11,0.5) rectangle (13,1.5);             
          \node[text=gray!3] at (12,0.95)          {\Large{\textbf{TOS}}};                  
          %\draw [line width=1ex, --] (13,1)  -- (14.5,1);                       
          \draw [rounded corners, line width=1ex, <-] (6,0.5) -- (6,0) -- (12,0) -- (12,0.5);
        \end{scope}

      \end{tikzpicture}
    }} &
    %\multicolumn{1}{m{28em}|}{
    \makecell[l]{
    \begin{minipage}[t]{\linewidth}%  
      For all other stack operations, which use any of the \gls{rotext} paths and for which the
      ``\texttt{TUCK}'' rule does not apply,
      the \gls{ps} must hold at least \textbf{three} values prior to the stack operation.
    \end{minipage}%
    } \\ \hline
    
  \end{longtable}
\end{center}  
\endgroup


%Interrupt extension
\subsection{Interrupt Extension}
\label{extensions:int}

Interrupt support is optional on the N1 processor.
If it is available, the Interupt Enable Register (IEN) (see \figref{extensions:int:ien:fig})
is mapped into the \gls{freg} space, allowing interrupts to be temporarily disabled.

\begin{figure}[H]
  \begin{center}
  \makebox[\textwidth][c]{
    %\scalebox{0.5125} {
    \scalebox{0.515} {
      \begin{tikzpicture}

        %Offset
        \node [align=left, anchor=west] at (-2,3) {
          \huge{Offset: \texttt{0x02}}
        };        
        %Index
        \node [above] at  (1,2) {15};
        \node [above] at  (3,2) {14};
        \node [above] at  (5,2) {13};
        \node [above] at  (7,2) {12};
        \node [above] at  (9,2) {11};
        \node [above] at (11,2) {10};
        \node [above] at (13,2)  {9};
        \node [above] at (15,2)  {8};
        \node [above] at (17,2)  {7};
        \node [above] at (19,2)  {6};
        \node [above] at (21,2)  {5};
        \node [above] at (23,2)  {4};
        \node [above] at (25,2)  {3};
        \node [above] at (27,2)  {2};
        \node [above] at (29,2)  {1};
        \node [above] at (31,2)  {0};
        %Read
        \node [align=right, anchor=east] at (0,1.5) {\large{Read}};
        \draw [thick, fill=white] (0,1) rectangle (32,2);
         %Write
        \node [align=right, anchor=east] at (0,0.5) {\large{Write}};
        \draw [thick, fill=white] (0,0) rectangle (32,1);
        %\draw [thick, fill=gray!3] (28,0) rectangle (30,1);  % 1
        %\node at (29,0.5) {\Large{\texttt{IMWE}}};
        \draw [thick, fill=gray!3] (0,0) rectangle (32,2);  % 0
        \node at (16,1)   {\Large{\texttt{IEN}}};
        \end{tikzpicture}
    }
  }
  \caption{Interrupt Enable Register}
  \label{extensions:int:ien:fig}
  \end{center}
\end{figure}

Interrupt  handling can be disabled through clearing the interrupt Enable Flag \texttt{IEN}.

The \texttt{IEN} bit is automatically cleared when an interrupt service routine is started
and must be set manually by writing a true value to the Interrupt Enable Register (see \tabref{extensions:int:ien:tab}).

\begingroup
\setlength{\LTleft}{-20cm plus -1fill}
\setlength{\LTright}{\LTleft}
\begin{center}
  \rowcolors{1}{gray!12}{white}                                         %set alternating row color
  \begin{longtable}{|c|c|c|c|}
    \rowcolor{white}
    \caption{Exception and Interrupt Mask Register Bit Description}
    \label{extensions:int:ien:tab} \\
    %Header
    \hline                                     
    \rowcolor{gray!25}
    \multicolumn{1}{|c|}{\textbf{\rule{0pt}{2.5ex}Bit}}       &  
    \multicolumn{1}{c|}{\textbf{\rule{0pt}{2.5ex}Postion}}    & 
    \multicolumn{1}{c|}{\textbf{\rule{0pt}{2.5ex}Description}} \\
     \hline
    \endhead                               
    %Footers
    \hline
    \rowcolor{white}
    \multicolumn{4}{r}{\tiny{...continued}} \\
    \endfoot
    \hline
    \endlastfoot

  
    %IEN
    \texttt{IEN} &
    15..0        &
    \multicolumn{1}{m{36em}|}{
      \makecell[l]{
        \begin{minipage}[t]{\linewidth}%  
          \centerline{\textbf{Interrupt Enable Flag}}
          \begin{description}[style=nextline]
          \item[Read:]
            \texttt{ true}: \tabto{4em} Interrupts are enabled \\
            \texttt{false}: \tabto{4em} Interrupts are disabled
          \item[Write:]
            \texttt{ true}: \tabto{4em} Enable interrupts \\
            \texttt{false}: \tabto{4em} Disable interrupts\\[1pt]
          \end{description}
        \end{minipage}%
    }}  \\ \hline
    
  \end{longtable}
\end{center}  
\endgroup

%KEY/EMIT extension
\subsection{KEY/EMIT Extension}
\label{extensions:key}

The KEY/EMIT extension adds support for an I/O device to the N1 core.
\gls{forth} words KEY, KEY?, EMIT, ands EMIT? dre direcly mapped to a set of \gls{freg}.

%KEY? register
\subsubsection{KEY? Register}
\label{extensions:key:keyq}

The KEY? register (see \figref{extensions:key:keyq:fig}) indicates whether there is data to be received from the input device. 

\begin{figure}[H]
  \begin{center}
  \makebox[\textwidth][c]{
    %\scalebox{0.5125} {
    \scalebox{0.515} {
      \begin{tikzpicture}

        %Offset
        \node [align=left, anchor=west] at (-2,3) {
          \huge{Offset: \texttt{0x04}}
        };        
        %Index
        \node [above] at  (1,2) {15};
        \node [above] at  (3,2) {14};
        \node [above] at  (5,2) {13};
        \node [above] at  (7,2) {12};
        \node [above] at  (9,2) {11};
        \node [above] at (11,2) {10};
        \node [above] at (13,2)  {9};
        \node [above] at (15,2)  {8};
        \node [above] at (17,2)  {7};
        \node [above] at (19,2)  {6};
        \node [above] at (21,2)  {5};
        \node [above] at (23,2)  {4};
        \node [above] at (25,2)  {3};
        \node [above] at (27,2)  {2};
        \node [above] at (29,2)  {1};
        \node [above] at (31,2)  {0};
        %Read
        \node [align=right, anchor=east] at (0,1.5) {\large{Read}};
        \draw [thick, fill=white] (0,1) rectangle (32,2);
         %Write
        \node [align=right, anchor=east] at (0,0.5) {\large{Write}};
        \draw [thick, fill=white] (0,0) rectangle (32,1);
        %\draw [thick, fill=gray!3] (28,0) rectangle (30,1);  % 1
        %\node at (29,0.5) {\Large{\texttt{IMWE}}};
        \draw [thick, fill=gray!3] (0,1) rectangle (32,2);  % 0
        \node at (16,1.5)   {\Large{\texttt{KEY?}}};
        \end{tikzpicture}
    }
  }
  \caption{KEY? Register}
  \label{extensions:key:keyq:fig}
  \end{center}
\end{figure}


The \texttt{KEY?} field in this read only register resembles the functionality of the \texttt{KEY?} word
(see \tabref{extensions:key:keyq:tab}.

\begingroup
\setlength{\LTleft}{-20cm plus -1fill}
\setlength{\LTright}{\LTleft}
\begin{center}
  \rowcolors{1}{gray!12}{white}                                         %set alternating row color
  \begin{longtable}{|c|c|c|c|}
    \rowcolor{white}
    \caption{Exception and Interrupt Mask Register Bit Description}
    \label{extensions:key:keyq:tab} \\
    %Header
    \hline                                     
    \rowcolor{gray!25}
    \multicolumn{1}{|c|}{\textbf{\rule{0pt}{2.5ex}Bit}}       &  
    \multicolumn{1}{c|}{\textbf{\rule{0pt}{2.5ex}Postion}}    & 
    \multicolumn{1}{c|}{\textbf{\rule{0pt}{2.5ex}Description}} \\
     \hline
    \endhead                               
    %Footers
    \hline
    \rowcolor{white}
    \multicolumn{4}{r}{\tiny{...continued}} \\
    \endfoot
    \hline
    \endlastfoot

    %KEY?
    \texttt{KEY?} &
    15..0        &
    \multicolumn{1}{m{36em}|}{
      \makecell[l]{
        \begin{minipage}[t]{\linewidth}%  
          \centerline{\textbf{Input Data Available Query}}
          \begin{description}[style=nextline]
          \item[Read:]
            \texttt{ true}: \tabto{4em} Input data is available \\
            \texttt{false}: \tabto{4em} Input data is not available \\[1pt]
          \end{description}
        \end{minipage}%
    }}  \\ \hline
    
  \end{longtable}
\end{center}  
\endgroup

%EMIT? register
\subsubsection{EMIT? Register}
\label{extensions:key:emitq}

The EMIT? (see \figref{extensions:key:emitq:fig}) register indicates whether the I/O device is ready to transmit data. 

\begin{figure}[H]
  \begin{center}
  \makebox[\textwidth][c]{
    %\scalebox{0.5125} {
    \scalebox{0.515} {
      \begin{tikzpicture}

        %Offset
        \node [align=left, anchor=west] at (-2,3) {
          \huge{Offset: \texttt{0x05}}
        };        
        %Index
        \node [above] at  (1,2) {15};
        \node [above] at  (3,2) {14};
        \node [above] at  (5,2) {13};
        \node [above] at  (7,2) {12};
        \node [above] at  (9,2) {11};
        \node [above] at (11,2) {10};
        \node [above] at (13,2)  {9};
        \node [above] at (15,2)  {8};
        \node [above] at (17,2)  {7};
        \node [above] at (19,2)  {6};
        \node [above] at (21,2)  {5};
        \node [above] at (23,2)  {4};
        \node [above] at (25,2)  {3};
        \node [above] at (27,2)  {2};
        \node [above] at (29,2)  {1};
        \node [above] at (31,2)  {0};
        %Read
        \node [align=right, anchor=east] at (0,1.5) {\large{Read}};
        \draw [thick, fill=white] (0,1) rectangle (32,2);
         %Write
        \node [align=right, anchor=east] at (0,0.5) {\large{Write}};
        \draw [thick, fill=white] (0,0) rectangle (32,1);
        %\draw [thick, fill=gray!3] (28,0) rectangle (30,1);  % 1
        %\node at (29,0.5) {\Large{\texttt{IMWE}}};
        \draw [thick, fill=gray!3] (0,1) rectangle (32,2);  % 0
        \node at (16,1.5)   {\Large{\texttt{EMIT?}}};
        \end{tikzpicture}
    }
  }
  \caption{EMIT? Register}
  \label{extensions:key:emitq:fig}
  \end{center}
\end{figure}

The \texttt{EMIT?} field in this read only register resembles the functionality of the \texttt{EMIT?} word
(see \tabref{extensions:key:emitq:tab}.

\begingroup
\setlength{\LTleft}{-20cm plus -1fill}
\setlength{\LTright}{\LTleft}
\begin{center}
  \rowcolors{1}{gray!12}{white}                                         %set alternating row color
  \begin{longtable}{|c|c|c|c|}
    \rowcolor{white}
    \caption{Exception and Interrupt Mask Register Bit Description}
    \label{extensions:key:keyq:tab} \\
    %Header
    \hline                                     
    \rowcolor{gray!25}
    \multicolumn{1}{|c|}{\textbf{\rule{0pt}{2.5ex}Bit}}       &  
    \multicolumn{1}{c|}{\textbf{\rule{0pt}{2.5ex}Postion}}    & 
    \multicolumn{1}{c|}{\textbf{\rule{0pt}{2.5ex}Description}} \\
     \hline
    \endhead                               
    %Footers
    \hline
    \rowcolor{white}
    \multicolumn{4}{r}{\tiny{...continued}} \\
    \endfoot
    \hline
    \endlastfoot

    %EMIT?
    \texttt{EMIT?} &
    15..0        &
    \multicolumn{1}{m{36em}|}{
      \makecell[l]{
        \begin{minipage}[t]{\linewidth}%  
          \centerline{\textbf{Output Device Ready Query}}
          \begin{description}[style=nextline]
          \item[Read:]
            \texttt{ true}: \tabto{4em} Output device is ready to transmit data \\
            \texttt{false}: \tabto{4em} Output device is not ready to transmit data \\[1pt]
          \end{description}
        \end{minipage}%
    }}  \\ \hline
    
  \end{longtable}
\end{center}  
\endgroup

%%PEEK register
%\subsubsection{Peek Register}
%\label{extensions:key:peek}
%
%The PEEK register (see \figref{extensions:key:peek:fig}) allows the software to preview at the input buffer of the I/O device,
%without removing any data. 
%This functionality is needed to implement the \texttt{KEY?} word.
%
%
%\begin{figure}[H]
%  \begin{center}
%  \makebox[\textwidth][c]{
%    %\scalebox{0.5125} {
%    \scalebox{0.515} {
%      \begin{tikzpicture}
%
%        %Offset
%        \node [align=left, anchor=west] at (-2,3) {
%          \huge{Offset: \texttt{0x06}}
%        };        
%        %Index
%        \node [above] at  (1,2) {15};
%        \node [above] at  (3,2) {14};
%        \node [above] at  (5,2) {13};
%        \node [above] at  (7,2) {12};
%        \node [above] at  (9,2) {11};
%        \node [above] at (11,2) {10};
%        \node [above] at (13,2)  {9};
%        \node [above] at (15,2)  {8};
%        \node [above] at (17,2)  {7};
%        \node [above] at (19,2)  {6};
%        \node [above] at (21,2)  {5};
%        \node [above] at (23,2)  {4};
%        \node [above] at (25,2)  {3};
%        \node [above] at (27,2)  {2};
%        \node [above] at (29,2)  {1};
%        \node [above] at (31,2)  {0};
%        %Read
%        \node [align=right, anchor=east] at (0,1.5) {\large{Read}};
%        \draw [thick, fill=white] (0,1) rectangle (32,2);
%         %Write
%        \node [align=right, anchor=east] at (0,0.5) {\large{Write}};
%        \draw [thick, fill=white] (0,0) rectangle (32,1);
%        %\draw [thick, fill=gray!3] (28,0) rectangle (30,1);  % 1
%        %\node at (29,0.5) {\Large{\texttt{IMWE}}};
%        \draw [thick, fill=gray!3] (0,1) rectangle (32,2);  % 0
%        \node at (16,1.5)   {\Large{\texttt{PEEK}}};
%        \end{tikzpicture}
%    }
%  }
%  \caption{EMIT? Register}
%  \label{extensions:key:peek:fig}
%  \end{center}
%\end{figure}
%
%The data in the \texttt{PEEK} field (see \tabref{extensions:key:emitq:tab} is only valid if
%\texttt{KEY?} holds a true value.
%
%\begingroup
%\setlength{\LTleft}{-20cm plus -1fill}
%\setlength{\LTright}{\LTleft}
%\begin{center}
%  \rowcolors{1}{gray!12}{white}                                         %set alternating row color
%  \begin{longtable}{|c|c|c|c|}
%    \rowcolor{white}
%    \caption{Exception and Interrupt Mask Register Bit Description}
%    \label{extensions:key:keyq:tab} \\
%    %Header
%    \hline                                     
%    \rowcolor{gray!25}
%    \multicolumn{1}{|c|}{\textbf{\rule{0pt}{2.5ex}Bit}}       &  
%    \multicolumn{1}{c|}{\textbf{\rule{0pt}{2.5ex}Postion}}    & 
%    \multicolumn{1}{c|}{\textbf{\rule{0pt}{2.5ex}Description}} \\
%     \hline
%    \endhead                               
%    %Footers
%    \hline
%    \rowcolor{white}
%    \multicolumn{4}{r}{\tiny{...continued}} \\
%    \endfoot
%    \hline
%    \endlastfoot
%
%    %PEEK
%    \texttt{PEEK} &
%    15..0        &
%    \multicolumn{1}{m{36em}|}{
%      \makecell[l]{
%        \begin{minipage}[t]{\linewidth}%  
%          \centerline{\textbf{Input Data Preview}}
%          \begin{description}[style=nextline]
%          \item[Read:]
%            Next word of read data from the input device \\[1pt]
%          \end{description}
%        \end{minipage}%
%    }}  \\ \hline
%    
%  \end{longtable}
%\end{center}  
%\endgroup

%KEY/EMIT register
\subsubsection{KEY/EMIT Register}
\label{extensions:key:key}

The KEY/EMIT register (see \figref{extensions:key:key:fig} is used to receive data from and to transmit daza to the I/O device.
Accesses to this register are blocking.

\begin{figure}[H]
  \begin{center}
  \makebox[\textwidth][c]{
    %\scalebox{0.5125} {
    \scalebox{0.515} {
      \begin{tikzpicture}

        %Offset
        \node [align=left, anchor=west] at (-2,3) {
          \huge{Offset: \texttt{0x06}}
        };        
        %Index
        \node [above] at  (1,2) {15};
        \node [above] at  (3,2) {14};
        \node [above] at  (5,2) {13};
        \node [above] at  (7,2) {12};
        \node [above] at  (9,2) {11};
        \node [above] at (11,2) {10};
        \node [above] at (13,2)  {9};
        \node [above] at (15,2)  {8};
        \node [above] at (17,2)  {7};
        \node [above] at (19,2)  {6};
        \node [above] at (21,2)  {5};
        \node [above] at (23,2)  {4};
        \node [above] at (25,2)  {3};
        \node [above] at (27,2)  {2};
        \node [above] at (29,2)  {1};
        \node [above] at (31,2)  {0};
        %Read
        \node [align=right, anchor=east] at (0,1.5) {\large{Read}};
        \draw [thick, fill=white] (0,1) rectangle (32,2);
         %Write
        \node [align=right, anchor=east] at (0,0.5) {\large{Write}};
        \draw [thick, fill=white] (0,0) rectangle (32,1);
        \draw [thick, fill=gray!3] (0,0) rectangle (32,1);  % 1
        \node at (16,0.5) {\Large{\texttt{EMIT}}};
        \draw [thick, fill=gray!3] (0,1) rectangle (32,2);  % 0
        \node at (16,1.5)   {\Large{\texttt{KEY}}};
        \end{tikzpicture}
    }
  }
  \caption{KEY/EMIT Register}
  \label{extensions:key:key:fig}
  \end{center}
\end{figure}

The \texttt{KEY} field resembles the \texttt{KEY} word and the \texttt{EMIT} field resembles the \texttt{EMIT} word
(see \tabref{extensions:key:key:tab}), 

\begingroup
\setlength{\LTleft}{-20cm plus -1fill}
\setlength{\LTright}{\LTleft}
\begin{center}
  \rowcolors{1}{gray!12}{white}                                         %set alternating row color
  \begin{longtable}{|c|c|c|c|}
    \rowcolor{white}
    \caption{Exception and Interrupt Mask Register Bit Description}
    \label{extensions:key:key:tab} \\
    %Header
    \hline                                     
    \rowcolor{gray!25}
    \multicolumn{1}{|c|}{\textbf{\rule{0pt}{2.5ex}Bit}}       &  
    \multicolumn{1}{c|}{\textbf{\rule{0pt}{2.5ex}Postion}}    & 
    \multicolumn{1}{c|}{\textbf{\rule{0pt}{2.5ex}Description}} \\
     \hline
    \endhead                               
    %Footers
    \hline
    \rowcolor{white}
    \multicolumn{4}{r}{\tiny{...continued}} \\
    \endfoot
    \hline
    \endlastfoot

    %KEY
    \texttt{KEY} &
    15..0        &
    \multicolumn{1}{m{36em}|}{
      \makecell[l]{
        \begin{minipage}[t]{\linewidth}%  
          \centerline{\textbf{Input Data}}
          \begin{description}[style=nextline]
          \item[Read:]
            Input data, removed from input device when read.\\[1pt]
          \end{description}
        \end{minipage}%
    }}  \\ \hline
    
    %EMIT
    \texttt{EMIT} &
    15..0        &
    \multicolumn{1}{m{36em}|}{
      \makecell[l]{
        \begin{minipage}[t]{\linewidth}%  
          \centerline{\textbf{Output Data}}
          \begin{description}[style=nextline]
          \item[Read:]
            Output data\\[1pt]
          \end{description}
        \end{minipage}%
    }}  \\ \hline
    
  \end{longtable}
\end{center}  
\endgroup

%%CATCH extension
%\subsection{CATCH Extension}
%\label{extensions:catch}
%
%The \gls{catchext} enhances the functionality of the Throw Code Register (see \secref{opcodes:freg:tcr}
%implement the \gls{forth} \mbox{\texttt{CATCH ( xt -- tc )}} instruction (see \tabref{extensions:catch:tcr:tab}).
%This extension includes two hidden register to preseve the parameter and the return stack depth at
%the time the \texttt{CATCH} instruction was executed.
%
%\begingroup
%\setlength{\LTleft}{-20cm plus -1fill}
%\setlength{\LTright}{\LTleft}
%\begin{center}
%  \rowcolors{1}{gray!12}{white}                                         %set alternating row color
%  \begin{longtable}{|c|c|c|c|}
%    \rowcolor{white}
%    \caption{Extended Throw Code Register Bit Description}
%    \label{extensions:catch:tcr:tab} \\
%    %Header
%    \hline                                     
%    \rowcolor{gray!25}
%    \multicolumn{1}{|c|}{\textbf{\rule{0pt}{2.5ex}Bit}}  &  
%    \multicolumn{1}{c|}{\textbf{\rule{0pt}{2.5ex}Postion}}    & 
%    \multicolumn{1}{c|}{\textbf{\rule{0pt}{2.5ex}Description}} \\
%     \hline
%    \endhead                               
%    %Footers
%    \hline
%    \rowcolor{white}
%    \multicolumn{4}{r}{\tiny{...continued}} \\
%    \endfoot
%    \hline
%    \endlastfoot
%
%    %TC
%    \texttt{TC} &
%    15..0        &
%    \multicolumn{1}{m{36em}|}{
%      \makecell[l]{
%        \begin{minipage}[t]{\linewidth}%  
%          \centerline{\textbf{Throw Code}}
%          \begin{description}[style=nextline]
%          \item[Read if \texttt{TC} $=$ \texttt{\$0000}:]
%            Read the latest \gls{tc} and
%            Pull previously preserved parameter and return stack depth from the \gls{rs}
%            into the hidden registers.
%            Trigger an underflow exception, if the \gls{rs} does not hold enouch content.
%          \item[Read if \texttt{TC} $\ne$ \texttt{\$0000}:]
%            Only read the latest \gls{tc} \\
%          \end{description}
%        \end{minipage}%
%    }}  \\ \hline
%
%    %THROW
%    \texttt{THROW} &
%    15..0        &
%    \multicolumn{1}{m{36em}|}{
%      \makecell[l]{
%        \begin{minipage}[t]{\linewidth}%  
%          \centerline{\textbf{Throw Trigger}}
%          \begin{description}[style=nextline]
%          \item[Write if \texttt{TC} $=$ \texttt{\$0000}:]
%            Store value \texttt{\$0000}, 
%            push the content of the hidden registers onto the \gls{rs}, and
%            capture the current parameter and return stack depth in the hidden registers.
%          \item[Write if \texttt{TC} $\ne$ \texttt{\$0000}:]
%            Store the \gls{tc},
%            restore the parameter and return stack depth as captured in the hidden registers,
%            pull the return address from the \gls{rs} and resume execution at this program location.
%            If no return address is availabble, resume at address \texttt{\$0000}.
%            Triggers an exception with the written \gls{tc}.\\
%          \end{description}
%        \end{minipage}%
%    }}  \\ \hline
%
%  \end{longtable}
%\end{center}  
%\endgroup
%
%The \texttt{CATCH} instruction is to be implemented as follows: \\
%                                           
%\begin{lstlisting}[frame=single, ]  % Start your code-block
%
%  :NONAME ( xt -- )    / inner CATCH sequence
%     0 LITERAL TCR!    / capture stack depths 
%     EXECUTE ;         / execute the xt
%  : CATCH ( xt -- tc )
%     [ COMPILE, ]      / call inner CATCH sequence
%     DROP TCR@ ;       / restore catch state
%
%\end{lstlisting}
