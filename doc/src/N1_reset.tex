%###############################################################################
%# N1 - Manual - Exceptions and Interrupts                                     #
%###############################################################################
%#    Copyright 2018 - 2019 Dirk Heisswolf                                     #
%#    This file is part of the N1 project.                                     #
%#                                                                             #
%#    N1 is free software: you can redistribute it and/or modify               #
%#    it under the terms of the GNU General Public License as published by     #
%#    the Free Software Foundation, either version 3 of the License, or        #
%#    (at your option) any later version.                                      #
%#                                                                             #
%#    N1 is distributed in the hope that it will be useful,                    #
%#    but WITHOUT ANY WARRANTY; without even the implied warranty of           #
%#    MERCHANTABILITY or FITNESS FOR A PARTICULAR PURPOSE.  See the            #
%#    GNU General Public License for more details.                             #
%#                                                                             #
%#    You should have received a copy of the GNU General Public License        #
%#    along with N1.  If not, see <http://www.gnu.org/licenses/>.              #
%###############################################################################
%# Version History:                                                            #
%#   March 4, 2019                                                             #
%#      - Initial release                                                      #
%###############################################################################

\section{Exceptions and Interrupts}
\label{excpt}

The N1 processor supports two mechanisms to stop the ongoing program flow in order
to react to an urgent hardware condition: Exceptions and Interrupts.

\subsection{Exceptions}
\label{excpt:general}
Exceptions are a mechanism to react to error conditions. Whenever an exception occurs,
the ongoing program flow is terminated and a handler routine, located at address \texttt{0x0000} 
is executed. Additionally, a \gls{tc} is pushed onto the \gls{ps} identifiies the cause of
problem. Based on this information, the handler can resore a save state of the systen



Resets are always triggerd at system startup, and may also occur during code execution,
triggered by the peripheral hardware.
When the N1 processor comes out of reset, it begins code execution at address
\texttt{0x0000}. The return stack is empty and the parameter stack holds one entry
of value \texttt{0x0000}. This stack entry indicates the reset condition to the startup code. 

\subsection{Exceptions}
\label{reset:excpt}
Exeptions are triggered by five hardware conditions. 


Whenever an exception event occurs




Additionally exceptions may be triggered by s


\subsection{Interrupts}
\label{reset:irq}
