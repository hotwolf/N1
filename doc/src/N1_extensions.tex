%###############################################################################
%# N1 - Manual - Instruction Set Extensions                                    #
%###############################################################################
%#    Copyright 2018 - 2019 Dirk Heisswolf                                     #
%#    This file is part of the N1 project.                                     #
%#                                                                             #
%#    N1 is free software: you can redistribute it and/or modify               #
%#    it under the terms of the GNU General Public License as published by     #
%#    the Free Software Foundation, either version 3 of the License, or        #
%#    (at your option) any later version.                                      #
%#                                                                             #
%#    N1 is distributed in the hope that it will be useful,                    #
%#    but WITHOUT ANY WARRANTY; without even the implied warranty of           #
%#    MERCHANTABILITY or FITNESS FOR A PARTICULAR PURPOSE.  See the            #
%#    GNU General Public License for more details.                             #
%#                                                                             #
%#    You should have received a copy of the GNU General Public License        #
%#    along with N1.  If not, see <http://www.gnu.org/licenses/>.              #
%###############################################################################
%# Version History:                                                            #
%#   October 11, 2019                                                          #
%#      - Initial release                                                      #
%###############################################################################

\section{Instruction Set Extensions}
\label{extensions}

The instruction set of the N1 processor (see \secref{opcodes}) reserves a number
of undefined \glspl{opcode} for functional extensions.
These extensions imply a trade-off between hardware complexity and functional 
improvements.
They can be selected individually for each system integrating the N1 processor
(see \secref{integration}).    

\subsection{ROT Extension}
\label{extensions:rot}

The \gls{rotext} adds two data paths to the \gls{us}, allowing direct data transfers
between the top and the third element of the \gls{ps}. These new stack transitions 
are performed by the regular stack instructions (see \secref{opcodes:stack}), using
some of the reserved stack transition patterns. 
\figref{extensions:rot:transpat} illustrates the usage of the \gls{rotext}. 

\begin{figure}[!h]
  %\begin{center}
  \makebox[\textwidth][c]{
    \scalebox{0.72} {
      \begin{tikzpicture}
        
        %Upwards
        \begin{scope}[shift={(0,14)}]
          
          %Stack instruction
          \draw [thick, fill=gray!3]  (1,4) rectangle (17,5);
          \draw [thick]               (1,4) rectangle  (2,5);
          \draw [thick]               (2,4) rectangle  (7,5); 
          \draw [thick, fill=white]   (7,4) rectangle  (8,5); 
          \draw [thick, fill=white]   (8,4) rectangle (10,5); 
          \draw [thick, fill=white]   (10,4) rectangle (12,5); 
          \draw [thick, fill=white]   (12,4) rectangle (13,5); 
          \draw [thick, fill=gray!48] (13,4) rectangle (14,5); 
          \draw [thick, fill=white]   (14,4) rectangle (15,5); 
          \draw [thick, fill=gray!48] (14,4) rectangle (15,5); 
          \draw [thick, fill=white]   (16,4) rectangle (17,5); 
          
          \node [above] at  (1.5,5) {15};
          \node [above] at  (2.5,5) {14};
          \node [above] at  (3.5,5) {13};
          \node [above] at  (4.5,5) {12};
          \node [above] at  (5.5,5) {11};
          \node [above] at  (6.5,5) {10};
          \node [above] at  (7.5,5) {9};
          \node [above] at  (8.5,5) {8};
          \node [above] at  (9.5,5) {7};
          \node [above] at (10.5,5) {6};
          \node [above] at (11.5,5) {5};
          \node [above] at (12.5,5) {4};
          \node [above] at (13.5,5) {3};
          \node [above] at (14.5,5) {2};
          \node [above] at (15.5,5) {1};
          \node [above] at (16.5,5) {0};
          
          \node         at (1.5,4.5)     {\huge{\texttt{;}}};
          \node         at (2.5,4.5)     {\huge{\texttt{0}}};
          \node         at (3.5,4.5)     {\huge{\texttt{0}}};
          \node         at (4.5,4.5)     {\huge{\texttt{0}}};
          \node         at (5.5,4.5)     {\huge{\texttt{0}}};   
          \node         at (6.5,4.5)     {\huge{\texttt{1}}};   
          
          \node         at (7.5,4.5)     {\Large{IST}};   
          \node         at (9,4.5)       {\Large{UST}};   
          \node         at (11,4.5)      {\Large{UST}};   
          \node         at (12.48,4.48)  {\small{UST}};
          \node         at (13.5,4.5)    {\huge{\texttt{\textbf{1}}}};             
          \node         at (15.48,4.48)  {\small{UST}};   
          \node         at (14.5,4.5)    {\huge{\texttt{\textbf{1}}}};             
          \node         at (16.5,4.5)    {\Large{IST}};   
          
          \node [below right] at (1,3.8) {\large{Stack Instruction}};
          
          %Bit field association
          \draw [ultra thick, dashed, ->]  (7.5,4)  -- (7.5,3)    -- (2.5,3)    -- (2.5,2);
          \draw [ultra thick, dashed, ->]  (9,4)    -- (9,2.8)    -- (5.5,2.8)  -- (5.5,2);
          \draw [ultra thick, dashed, ->]  (11,4)   -- (11,2.6)   -- (8.5,2.6)  -- (8.5,2);
          \draw [ultra thick, dashed, ->]  (12.5,4) -- (12.5,2.4) -- (11.5,2.4) -- (11.5,2);
          \draw [ultra thick, dashed, ->]  (15.5,4) -- (15.5,2.4) -- (15,2.4)   -- (15,2);
          \draw [ultra thick, dashed, ->]  (16.5,4) -- (16.5,2.4) -- (18,2.4)   -- (18,2);
          
          %Lower parameter stack
          \draw [thick, fill=gray!3]  (1,1) -- (2,1) -- (2,2) -- (1,2);
          \draw [ultra thick, ->]     (2,1.8) --  (3,1.8);
          \draw [ultra thick, <-]     (2,1.2) --  (3,1.2);         
          
          %Upper parameter stack
          \draw [thick, fill=gray!3]  (3,1) rectangle  (5,2);
          \draw [ultra thick, ->]     (5,1.8) --  (6,1.8);
          \draw [ultra thick, <->]    (5,1.5) --  (6,1.5);
          \draw [ultra thick, <-]     (5,1.2) --  (6,1.2);         
          
          \draw [thick, fill=gray!3]  (6,1) rectangle  (8,2);
          \draw [ultra thick, ->]     (8,1.8) --  (9,1.8);
          \draw [ultra thick, <->]    (8,1.5) --  (9,1.5);
          \draw [ultra thick, <-]     (8,1.2) --  (9,1.2);         
          
          \draw [thick, fill=gray!3]  (9,1) rectangle  (11,2);
          %\draw [ultra thick, ->]     (11,1.8) --  (12,1.8);
          %\draw [ultra thick, <->]    (11,1.5) --  (12,1.5);
          \draw [ultra thick, <-]     (11,1.2) --  (12,1.2);         

          \draw [thick, fill=gray!3]  (12,1) rectangle  (14,2);
          \node at (13,1.45)          {\Large{\textbf{TOS}}};        
          \node at (3.5,0.5)          {\large{Parameter stack}};

          %ROT extension
          \draw [rounded corners, ultra thick, ->] (7,1) -- (7,0.5) -- (13,0.5) -- (13,1);

          %Stack boundary
          \draw [ultra thick, ->]     (14,1.8)    --  (15.5,1.8);
          %\draw [ultra thick, <->]    (14,1.5)    --  (15.5,1.5);
          %\draw [ultra thick, <-]     (14,1.2)    --  (15.5,1.2);         
          \draw [dotted]              (14.75,2.2) --  (14.75,0.2);
          
          %Upper return stack
          \draw [thick, fill=gray!24] (15.5,1) rectangle  (17.5,2);
          \node at (16.5,1.45)        {\Large{\textbf{TOS}}};
          \draw [ultra thick, ->]     (17.5,1.8) --  (18.5,1.8);
          \draw [ultra thick, <-]     (17.5,1.2) --  (18.5,1.2);         
          \node at (17.5,0.5)         {\large{Return stack}};
          
          %Lower return stack
          \draw [thick, fill=gray!24] (19.5,1) -- (18.5,1) -- (18.5,2) -- (19.5,2);
          
        \end{scope}
       
        %Downwards        
        \begin{scope}[shift={(0,7)}]
          
          %Stack instruction
          \draw [thick, fill=gray!3]   (1,4) rectangle (17,5);
          \draw [thick]                (1,4) rectangle  (2,5);
          \draw [thick]                (2,4) rectangle  (7,5); 
          \draw [thick, fill=white]    (7,4) rectangle  (8,5); 
          \draw [thick, fill=white]    (8,4) rectangle  (9,5); 
          \draw [thick, fill=gray!48]  (9,4) rectangle (10,5); 
          \draw [thick, fill=gray!48] (10,4) rectangle (11,5); 
          \draw [thick, fill=white]   (11,4) rectangle (12,5); 
          \draw [thick, fill=white]   (12,4) rectangle (14,5); 
          \draw [thick, fill=white]   (14,4) rectangle (16,5); 
          \draw [thick, fill=white]   (16,4) rectangle (17,5); 
          
          \node [above] at  (1.5,5) {15};
          \node [above] at  (2.5,5) {14};
          \node [above] at  (3.5,5) {13};
          \node [above] at  (4.5,5) {12};
          \node [above] at  (5.5,5) {11};
          \node [above] at  (6.5,5) {10};
          \node [above] at  (7.5,5) {9};
          \node [above] at  (8.5,5) {8};
          \node [above] at  (9.5,5) {7};
          \node [above] at (10.5,5) {6};
          \node [above] at (11.5,5) {5};
          \node [above] at (12.5,5) {4};
          \node [above] at (13.5,5) {3};
          \node [above] at (14.5,5) {2};
          \node [above] at (15.5,5) {1};
          \node [above] at (16.5,5) {0};
          
          \node         at (1.5,4.5)     {\huge{\texttt{;}}};
          \node         at (2.5,4.5)     {\huge{\texttt{0}}};
          \node         at (3.5,4.5)     {\huge{\texttt{0}}};
          \node         at (4.5,4.5)     {\huge{\texttt{0}}};
          \node         at (5.5,4.5)     {\huge{\texttt{0}}};   
          \node         at (6.5,4.5)     {\huge{\texttt{1}}};   
          
          \node         at (7.5,4.5)     {\Large{IST}};   
          \node         at (8.48,4.48)   {\small{UST}};
          \node         at (9.5,4.5)     {\huge{\texttt{\textbf{1}}}};             
          \node         at (11.48,4.48)  {\small{UST}};
          \node         at (10.5,4.5)    {\huge{\texttt{\textbf{1}}}};             
          \node         at (13,4.5)      {\Large{UST}};   
          \node         at (15,4.5)      {\Large{UST}};   
          \node         at (16.5,4.5)    {\Large{IST}};   
          
          \node [below right] at (1,3.8) {\large{Stack Instruction}};
          
          %Bit field association
          \draw [ultra thick, dashed, ->]  (7.5,4)  -- (7.5,3)    -- (2.5,3)    -- (2.5,2);
          \draw [ultra thick, dashed, ->]  (8.5,4)  -- (8.5,2.8)  -- (5.5,2.8)  -- (5.5,2);
          \draw [ultra thick, dashed, ->]  (11.5,4) -- (11.5,2.6) -- (8.5,2.6)  -- (8.5,2);
          \draw [ultra thick, dashed, ->]  (13,4)   -- (13,2.4)   -- (11.5,2.4) -- (11.5,2);
          \draw [ultra thick, dashed, ->]  (15,4)   -- (15,2);
          \draw [ultra thick, dashed, ->]  (16.5,4) -- (16.5,2.4) -- (18,2.4) -- (18,2);
          
          %Lower parameter stack
          \draw [thick, fill=gray!3]  (1,1) -- (2,1) -- (2,2) -- (1,2);
          \draw [ultra thick, ->]     (2,1.8) --  (3,1.8);
          \draw [ultra thick, <-]     (2,1.2) --  (3,1.2);         
          
          %Upper parameter stack
          \draw [thick, fill=gray!3]  (3,1) rectangle  (5,2);
          %\draw [ultra thick, ->]     (5,1.8) --  (6,1.8);
          %\draw [ultra thick, <->]    (5,1.5) --  (6,1.5);
          \draw [ultra thick, <-]     (5,1.2) --  (6,1.2);         
          
          \draw [thick, fill=gray!3]  (6,1) rectangle  (8,2);
          \draw [ultra thick, ->]     (8,1.8) --  (9,1.8);
          %\draw [ultra thick, <->]    (8,1.5) --  (9,1.5);
          %\draw [ultra thick, <-]     (8,1.2) --  (9,1.2);         
          
          \draw [thick, fill=gray!3]  (9,1) rectangle  (11,2);
          \draw [ultra thick, ->]     (11,1.8) --  (12,1.8);
          \draw [ultra thick, <->]    (11,1.5) --  (12,1.5);
          \draw [ultra thick, <-]     (11,1.2) --  (12,1.2);         
          
          \draw [thick, fill=gray!3]  (12,1) rectangle  (14,2);
          \node at (13,1.45)          {\Large{\textbf{TOS}}};        
          \node at (3.5,0.5)           {\large{Parameter stack}};

          %ROT extension
          \draw [rounded corners, ultra thick, <-] (7,1) -- (7,0.5) -- (13,0.5) -- (13,1);
         
          %Stack boundary
          \draw [ultra thick, ->]     (14,1.8)    --  (15.5,1.8);
          \draw [ultra thick, <->]    (14,1.5)    --  (15.5,1.5);
          \draw [ultra thick, <-]     (14,1.2)    --  (15.5,1.2);         
          \draw [dotted]              (14.75,2.2) --  (14.75,0.2);
          
          %Upper return stack
          \draw [thick, fill=gray!24] (15.5,1) rectangle  (17.5,2);
          \node at (16.5,1.45)        {\Large{\textbf{TOS}}};
          \draw [ultra thick, ->]     (17.5,1.8) --  (18.5,1.8);
          \draw [ultra thick, <-]     (17.5,1.2) --  (18.5,1.2);         
          \node at (17.5,0.5)         {\large{Return stack}};
          
          %Lower return stack
          \draw [thick, fill=gray!24] (19.5,1) -- (18.5,1) -- (18.5,2) -- (19.5,2);

        \end{scope}
        
        %Exchange        
        \begin{scope}
          
          %Stack instruction
          \draw [thick, fill=gray!3]  (1,4) rectangle (17,5);
          \draw [thick]               (1,4) rectangle  (2,5);
          \draw [thick]               (2,4) rectangle  (7,5); 
          \draw [thick, fill=white]   (7,4) rectangle  (8,5); 
          \draw [thick, fill=white]   (8,4) rectangle  (9,5); 
          \draw [thick, fill=gray!48]  (9,4) rectangle (10,5); 
          \draw [thick, fill=gray!48] (10,4) rectangle (11,5); 
          \draw [thick, fill=white]   (11,4) rectangle (12,5); 
          \draw [thick, fill=white]   (12,4) rectangle (13,5); 
          \draw [thick, fill=gray!48] (13,4) rectangle (14,5); 
          \draw [thick, fill=white]   (14,4) rectangle (15,5); 
          \draw [thick, fill=gray!48] (14,4) rectangle (15,5); 
          \draw [thick, fill=white]   (16,4) rectangle (17,5); 
        
          \node [above] at  (1.5,5) {15};
          \node [above] at  (2.5,5) {14};
          \node [above] at  (3.5,5) {13};
          \node [above] at  (4.5,5) {12};
          \node [above] at  (5.5,5) {11};
          \node [above] at  (6.5,5) {10};
          \node [above] at  (7.5,5) {9};
          \node [above] at  (8.5,5) {8};
          \node [above] at  (9.5,5) {7};
          \node [above] at (10.5,5) {6};
          \node [above] at (11.5,5) {5};
          \node [above] at (12.5,5) {4};
          \node [above] at (13.5,5) {3};
          \node [above] at (14.5,5) {2};
          \node [above] at (15.5,5) {1};
          \node [above] at (16.5,5) {0};
          
          \node         at (1.5,4.5)     {\huge{\texttt{;}}};
          \node         at (2.5,4.5)     {\huge{\texttt{0}}};
          \node         at (3.5,4.5)     {\huge{\texttt{0}}};
          \node         at (4.5,4.5)     {\huge{\texttt{0}}};
          \node         at (5.5,4.5)     {\huge{\texttt{0}}};   
          \node         at (6.5,4.5)     {\huge{\texttt{1}}};   
          
          \node         at (7.5,4.5)     {\Large{IST}};   
          \node         at (8.48,4.48)   {\small{UST}};
          \node         at (9.5,4.5)     {\huge{\texttt{\textbf{1}}}};             
          \node         at (11.48,4.48)  {\small{UST}};
          \node         at (10.5,4.5)    {\huge{\texttt{\textbf{1}}}};             

          \node         at (12.48,4.48)  {\small{UST}};
          \node         at (13.5,4.5)    {\huge{\texttt{\textbf{1}}}};             
          \node         at (15.48,4.48)  {\small{UST}};   
          \node         at (14.5,4.5)    {\huge{\texttt{\textbf{1}}}};             
          \node         at (16.5,4.5)    {\Large{IST}};   

          %\node         at (13,4.5)      {\Large{UST}};   
          %\node         at (15,4.5)      {\Large{UST}};   
          %\node         at (16.5,4.5)    {\Large{IST}};   
          
          \node [below right] at (1,3.8) {\large{Stack Instruction}};
          
          %Bit field association
          \draw [ultra thick, dashed, ->]  (7.5,4)  -- (7.5,3)    -- (2.5,3)    -- (2.5,2);
          \draw [ultra thick, dashed, ->]  (8.5,4)  -- (8.5,2.8)  -- (5.5,2.8)  -- (5.5,2);
          \draw [ultra thick, dashed, ->]  (11.5,4) -- (11.5,2.6) -- (8.5,2.6)  -- (8.5,2);
          \draw [ultra thick, dashed, ->]  (12.5,4) -- (12.5,2.4) -- (11.5,2.4) -- (11.5,2);
          \draw [ultra thick, dashed, ->]  (15.5,4) -- (15.5,2.4) -- (15,2.4)   -- (15,2);
          \draw [ultra thick, dashed, ->]  (16.5,4) -- (16.5,2.4) -- (18,2.4) -- (18,2);
          
          %Lower parameter stack
          \draw [thick, fill=gray!3]  (1,1) -- (2,1) -- (2,2) -- (1,2);
          \draw [ultra thick, ->]     (2,1.8) --  (3,1.8);
          \draw [ultra thick, <-]     (2,1.2) --  (3,1.2);         
          
          %Upper parameter stack
          \draw [thick, fill=gray!3]  (3,1) rectangle  (5,2);
          %\draw [ultra thick, ->]     (5,1.8) --  (6,1.8);
          %\draw [ultra thick, <->]    (5,1.5) --  (6,1.5);
          \draw [ultra thick, <-]     (5,1.2) --  (6,1.2);         
          
          \draw [thick, fill=gray!3]  (6,1) rectangle  (8,2);
          \draw [ultra thick, ->]     (8,1.8) --  (9,1.8);
          %\draw [ultra thick, <->]    (8,1.5) --  (9,1.5);
          %\draw [ultra thick, <-]     (8,1.2) --  (9,1.2);         
          
          \draw [thick, fill=gray!3]  (9,1) rectangle  (11,2);
          %\draw [ultra thick, ->]     (11,1.8) --  (12,1.8);
          %\draw [ultra thick, <->]    (11,1.5) --  (12,1.5);
          \draw [ultra thick, <-]     (11,1.2) --  (12,1.2);         
          
          \draw [thick, fill=gray!3]  (12,1) rectangle  (14,2);
          \node at (13,1.45)          {\Large{\textbf{TOS}}};        
          \node at (3.5,0.5)           {\large{Parameter stack}};

          %ROT extension
          \draw [rounded corners, ultra thick, <->] (7,1) -- (7,0.5) -- (13,0.5) -- (13,1);
         
          %Stack boundary
          \draw [ultra thick, ->]     (14,1.8)    --  (15.5,1.8);
          %\draw [ultra thick, <->]    (14,1.5)    --  (15.5,1.5);
          %\draw [ultra thick, <-]     (14,1.2)    --  (15.5,1.2);         
          \draw [dotted]              (14.75,2.2) --  (14.75,0.2);
          
          %Upper return stack
          \draw [thick, fill=gray!24] (15.5,1) rectangle  (17.5,2);
          \node at (16.5,1.45)        {\Large{\textbf{TOS}}};
          \draw [ultra thick, ->]     (17.5,1.8) --  (18.5,1.8);
          \draw [ultra thick, <-]     (17.5,1.2) --  (18.5,1.2);         
          \node at (17.5,0.5)         {\large{Return stack}};
          
          %Lower return stack
          \draw [thick, fill=gray!24] (19.5,1) -- (18.5,1) -- (18.5,2) -- (19.5,2);

        \end{scope}
        
      \end{tikzpicture}
    }
  }
  \caption{Stack transitions of the \gls{rotext}}
  \label{extensions:rot:transpat}
  %\end{center}
\end{figure}

\subsubsection{Accellerated Stack Operations}
\label{extensions:rot:ops}

The \gls{rotext} improves the execution time and code density of the three commmon
stack operations \texttt{TUCK}, \texttt{ROT}, and \texttt{-ROT}
(see \tabref{extensions:rot:mapping}). 
This means that all common single-cell \gls{ps} operations shown in
\tabref{opcodes:stack:mapping} can be executed in one cycle if the \gls{rotext} is
enabled.

\begingroup
\setlength{\LTleft}{-20cm plus -1fill}
\setlength{\LTright}{\LTleft}
\begin{center}
  \rowcolors{1}{gray!12}{white}                                         %set alternating row color
  \begin{longtable}{|c|c|c|c|}
    \rowcolor{white}
    \caption{Improved stack operations}
    \label{extensions:rot:mapping} \\
    %Header
    \hline                                     
    \rowcolor{gray!25}
    \multicolumn{1}{|c|}{\textbf{\rule{0pt}{2.5ex}Word}}       &  
    \multicolumn{1}{c|}{\textbf{\rule{0pt}{2.5ex}Description}} & 
    \multicolumn{1}{c|}{\textbf{\rule{0pt}{2.5ex}Transitions}} & 
    \multicolumn{1}{c|}{\textbf{\rule{0pt}{2.5ex}Opcode}} \\
    \hline
    \endhead                               
    %Footers
    \hline
    \rowcolor{white}
    \multicolumn{4}{r}{\tiny{...continued}} \\
    \endfoot
    \hline
    \endlastfoot

    %TUCK
    \texttt{TUCK} &
    ( x1 x2 -- x2 x1 x2 ) &
    \multicolumn{1}{m{21.35em}|}{
    \scalebox{0.4} {
      \begin{tikzpicture}
        \begin{scope}[shift={(0,0)}]
          \draw [thick, fill=gray!3]  (0,0.5) -- (1,0.5) -- (1,1.5) -- (0,1.5);%
          \draw [line width=1ex, <-] (1,1) -- (2,1);                           %
          \draw [thick, fill=gray!3]  (2,0.5) rectangle (4,1.5);               %PS+3
          \draw [line width=1ex, <-] (4,1) -- (5,1);                           %
          \draw [thick, fill=gray!3]  (5,0.5) rectangle (7,1.5);               %PS+2
          %\draw [line width=1ex, <->] (7,1) -- (8,1);                         %
          \draw [thick, fill=gray!3]  (8,0.5) rectangle (10,1.5);              %PS+1
          %\draw [line width=1ex, --] (10,1) -- (11,1);                        %
          \draw [thick, fill=gray!3]  (11,0.5) rectangle (13,1.5);             %PS TOS
          \node at (12,0.95)          {\Large{\textbf{TOS}}};                  %
          %\draw [line width=1ex, --] (13,1)  -- (14.5,1);                     %
          \draw [rounded corners, line width=1ex, <-] (6,0.5) -- (6,0) -- (12,0) -- (12,0.5);
          \draw [dotted]              (13.75,0.5) -- (13.75,1.7);              %
          \draw [thick, fill=gray!24] (14.5,0.5) rectangle (16.5,1.5);         %RS TOS
          \node at (15.5,0.95)        {\Large{\textbf{TOS}}};                  %
          %\draw [line width=1ex, --] (16.5,1) -- (17.5,1);                    % 
          \draw [thick, fill=gray!24] (18.5,0.5) -- (17.5,0.5) -- (17.5,1.5) -- (18.5,1.5);
        \end{scope}
       \end{tikzpicture}
    }} &
    \multicolumn{1}{m{4.25em}|}{
    \makecell[c]{ 
      \texttt{0x07C0}
    }} \\ \hline

    %ROT
    \texttt{ROT} &
    ( x1 x2 x3 -- x2 x3 x1 ) &
    \multicolumn{1}{m{21.35em}|}{
    \scalebox{0.4} {
      \begin{tikzpicture}
        \begin{scope}[shift={(0,0)}]
          \draw [thick, fill=gray!3]  (0,0.5) -- (1,0.5) -- (1,1.5) -- (0,1.5);%
          %\draw [line width=1ex, --] (1,1) -- (2,1);                          %
          \draw [thick, fill=gray!3]  (2,0.5) rectangle (4,1.5);               %PS+3
          %\draw [line width=1ex, --] (4,1) -- (5,1);                          %
          \draw [thick, fill=gray!3]  (5,0.5) rectangle (7,1.5);               %PS+2
          \draw [line width=1ex, <-]  (7,1) -- (8,1);                          %
          \draw [thick, fill=gray!3]  (8,0.5) rectangle (10,1.5);              %PS+1
          \draw [line width=1ex, <-]  (10,1) -- (11,1);                        %
          \draw [thick, fill=gray!3]  (11,0.5) rectangle (13,1.5);             %PS TOS
          \node at (12,0.95)          {\Large{\textbf{TOS}}};                  %
          %\draw [line width=1ex, --] (13,1)  -- (14.5,1);                     %
          \draw [rounded corners, line width=1ex, ->] (6,0.5) -- (6,0) -- (12,0) -- (12,0.5);
          \draw [dotted]              (13.75,0.5) -- (13.75,1.7);              %
          \draw [thick, fill=gray!24] (14.5,0.5) rectangle (16.5,1.5);         %RS TOS
          \node at (15.5,0.95)        {\Large{\textbf{TOS}}};                  %
          %\draw [line width=1ex, --] (16.5,1) -- (17.5,1);                    % 
          \draw [thick, fill=gray!24] (18.5,0.5) -- (17.5,0.5) -- (17.5,1.5) -- (18.5,1.5);
        \end{scope}
      \end{tikzpicture}
    }} &
    \multicolumn{1}{m{4.25em}|}{
    \makecell[c]{ 
      \texttt{0x041C}
    }} \\ \hline
    
    %-ROT
    \texttt{-ROT} &
    ( x1 x2 x3 -- x3 x1 x2 ) &
    \multicolumn{1}{m{21.35em}|}{
    \scalebox{0.4} {
      \begin{tikzpicture}
        \begin{scope}[shift={(0,0)}]
          \draw [thick, fill=gray!3]  (0,0.5) -- (1,0.5) -- (1,1.5) -- (0,1.5);%
          %\draw [line width=1ex, --] (1,1) -- (2,1);                          %
          \draw [thick, fill=gray!3]  (2,0.5) rectangle (4,1.5);               %PS+3
          %\draw [line width=1ex, --] (4,1) -- (5,1);                          %
          \draw [thick, fill=gray!3]  (5,0.5) rectangle (7,1.5);               %PS+2
          \draw [line width=1ex, ->]  (7,1) -- (8,1);                          %
          \draw [thick, fill=gray!3]  (8,0.5) rectangle (10,1.5);              %PS+1
          \draw [line width=1ex, ->]  (10,1) -- (11,1);                        %
          \draw [thick, fill=gray!3]  (11,0.5) rectangle (13,1.5);             %PS TOS
          \node at (12,0.95)          {\Large{\textbf{TOS}}};                  %
          %\draw [line width=1ex, --] (13,1)  -- (14.5,1);                     %
          \draw [rounded corners, line width=1ex, <-] (6,0.5) -- (6,0) -- (12,0) -- (12,0.5);
          \draw [dotted]              (13.75,0.5) -- (13.75,1.7);              %
          \draw [thick, fill=gray!24] (14.5,0.5) rectangle (16.5,1.5);         %RS TOS
          \node at (15.5,0.95)        {\Large{\textbf{TOS}}};                  %
          %\draw [line width=1ex, --] (16.5,1) -- (17.5,1);                    % 
          \draw [thick, fill=gray!24] (18.5,0.5) -- (17.5,0.5) -- (17.5,1.5) -- (18.5,1.5);
        \end{scope}
      \end{tikzpicture}
    }} &
    \multicolumn{1}{m{4.25em}|}{
    \makecell[c]{ 
      \texttt{0x04E0}
    }} \\ \hline

  \end{longtable}
\end{center}  
\endgroup

N1 processors with \gls{rotext} are backward compatible to the ones without.
All stack operations can still be executed as listed in \tabref{opcodes:stack:mapping},
even if the \gls{rotext} is enabled.

\subsubsection{Stack Underflow Detection}
\label{opcodes:stack:uf}

The \gls{rotext} introduces three new stack underflow detection rules.
These rules are listed in \tabref{opcodes:stack:ufrules}.

\begingroup
\setlength{\LTleft}{-20cm plus -1fill}
\setlength{\LTright}{\LTleft}
\begin{center}
  \rowcolors{1}{gray!12}{white}                                         %set alternating row color
  \begin{longtable}{|c|c|c|}
    \rowcolor{white}
    \caption{Rules of Stack Underflow Detection}
    \label{opcodes:stack:ufrules} \\
    %Header
    \hline                                     
    \rowcolor{gray!25}
    \multicolumn{1}{|c|}{\textbf{\rule{0pt}{2.5ex}Rule}}       &  
    \multicolumn{1}{c|}{\textbf{\rule{0pt}{2.5ex}Transitions}} & 
    \multicolumn{1}{c|}{\textbf{\rule{0pt}{2.5ex}Description}} \\
    \hline
    \endhead                               
    %Footers
    \hline
    \rowcolor{white}
    \multicolumn{3}{r}{\tiny{...continued}} \\
    \endfoot
    \hline
    \endlastfoot

    %TUCK rule
    \multicolumn{1}{|m{3em}|}{
    \makecell[c]{ 
      ``\texttt{TUCK}''\\
      Rule}} &
    \multicolumn{1}{m{11.5em}|}{
    \scalebox{0.4} {
      \begin{tikzpicture}
        \begin{scope}[shift={(0,0)}]
          \draw [thick, fill=gray!3]  (3,0.5) -- (4,0.5) -- (4,1.5) -- (3,1.5);
          \draw [line width=1ex, <-] (4,1) -- (5,1);                           
          \draw [thick, fill=gray!3]  (5,0.5) rectangle (7,1.5);               
          %\draw [line width=1ex, <->] (7,1) -- (8,1);                         
          \draw [thick, fill=gray]  (8,0.5) rectangle (10,1.5);              
          %\draw [line width=1ex, --] (10,1) -- (11,1);                        
          \draw [thick, fill=gray]  (11,0.5) rectangle (13,1.5);             
          \node[text=gray!3] at (12,0.95)          {\Large{\textbf{TOS}}};                  
          %\draw [line width=1ex, --] (13,1)  -- (14.5,1);                       
          \draw [rounded corners, line width=1ex, <-] (6,0.5) -- (6,0) -- (12,0) -- (12,0.5);
        \end{scope}
      \end{tikzpicture}
    }} &
    %\multicolumn{1}{m{28em}|}{
    \makecell[l]{
    \begin{minipage}[t]{\linewidth}%  
      If the downward \gls{rotext} path is used and the target cell is shifted further downward,
      then the \gls{ps} must hold at least \textbf{two} values prior to the stack operation.
    \end{minipage}%
    } \\ \hline

    %TUCK rule
    \multicolumn{1}{|m{3em}|}{
    \makecell[c]{ 
      ``\texttt{ROT}''\\
      Rule}} &
    \multicolumn{1}{m{11.5em}|}{
    \scalebox{0.4} {
      \begin{tikzpicture}
        \path []  (3,8)   -- (3,9.7);
        
        \begin{scope}[shift={(0,0)}]
          \draw [thick, fill=gray!3]  (3,0.5) -- (4,0.5) -- (4,1.5) -- (3,1.5);
          \draw [line width=1ex, ->] (4,1) -- (5,1);                           
          \draw [thick, fill=gray]  (5,0.5) rectangle (7,1.5);               
          %\draw [line width=1ex, <->] (7,1) -- (8,1);                         
          \draw [thick, fill=gray]  (8,0.5) rectangle (10,1.5);              
          %\draw [line width=1ex, --] (10,1) -- (11,1);                        
          \draw [thick, fill=gray]  (11,0.5) rectangle (13,1.5);             
          \node[text=gray!3] at (12,0.95)          {\Large{\textbf{TOS}}};                  
          %\draw [line width=1ex, --] (13,1)  -- (14.5,1);                       
          \draw [rounded corners, line width=1ex, ->] (6,0.5) -- (6,0) -- (12,0) -- (12,0.5);
        \end{scope}

        \begin{scope}[shift={(0,2)}]
          \draw [thick, fill=gray!3]  (3,0.5) -- (4,0.5) -- (4,1.5) -- (3,1.5);
          %\draw [line width=1ex, <-] (4,1) -- (5,1);                           
          \draw [line width=1ex, -|] (4,1) -- (4.4,1);
          \draw [line width=1ex, |-] (4.6,1) -- (5,1);
          \draw [thick, fill=gray]  (5,0.5) rectangle (7,1.5);               
          %\draw [line width=1ex, <->] (7,1) -- (8,1);                         
          \draw [thick, fill=gray]  (8,0.5) rectangle (10,1.5);              
          %\draw [line width=1ex, --] (10,1) -- (11,1);                        
          \draw [thick, fill=gray]  (11,0.5) rectangle (13,1.5);             
          \node[text=gray!3] at (12,0.95)          {\Large{\textbf{TOS}}};                  
          %\draw [line width=1ex, --] (13,1)  -- (14.5,1);                       
          \draw [rounded corners, line width=1ex, ->] (6,0.5) -- (6,0) -- (12,0) -- (12,0.5);
        \end{scope}

        \begin{scope}[shift={(0,4)}]
          \draw [thick, fill=gray!3]  (3,0.5) -- (4,0.5) -- (4,1.5) -- (3,1.5);
          \draw [line width=1ex, <-] (4,1) -- (5,1);                           
          \draw [thick, fill=gray]  (5,0.5) rectangle (7,1.5);               
          %\draw [line width=1ex, <->] (7,1) -- (8,1);                         
          \draw [thick, fill=gray]  (8,0.5) rectangle (10,1.5);              
          %\draw [line width=1ex, --] (10,1) -- (11,1);                        
          \draw [thick, fill=gray]  (11,0.5) rectangle (13,1.5);             
          \node[text=gray!3] at (12,0.95)          {\Large{\textbf{TOS}}};                  
          %\draw [line width=1ex, --] (13,1)  -- (14.5,1);                       
          \draw [rounded corners, line width=1ex, ->] (6,0.5) -- (6,0) -- (12,0) -- (12,0.5);
        \end{scope}

        \begin{scope}[shift={(0,6)}]
          \draw [thick, fill=gray!3]  (3,0.5) -- (4,0.5) -- (4,1.5) -- (3,1.5);
          \draw [line width=1ex, ->] (4,1) -- (5,1);                           
          \draw [thick, fill=gray]  (5,0.5) rectangle (7,1.5);               
          %\draw [line width=1ex, <->] (7,1) -- (8,1);                         
          \draw [thick, fill=gray]  (8,0.5) rectangle (10,1.5);              
          %\draw [line width=1ex, --] (10,1) -- (11,1);                        
          \draw [thick, fill=gray]  (11,0.5) rectangle (13,1.5);             
          \node[text=gray!3] at (12,0.95)          {\Large{\textbf{TOS}}};                  
          %\draw [line width=1ex, --] (13,1)  -- (14.5,1);                       
          \draw [rounded corners, line width=1ex, <-] (6,0.5) -- (6,0) -- (12,0) -- (12,0.5);
        \end{scope}
        
        \begin{scope}[shift={(0,8)}]
          \draw [thick, fill=gray!3]  (3,0.5) -- (4,0.5) -- (4,1.5) -- (3,1.5);
          %\draw [line width=1ex, <-] (4,1) -- (5,1);                           
          \draw [line width=1ex, -|] (4,1) -- (4.4,1);
          \draw [line width=1ex, |-] (4.6,1) -- (5,1);
          \draw [thick, fill=gray]  (5,0.5) rectangle (7,1.5);               
          %\draw [line width=1ex, <->] (7,1) -- (8,1);                         
          \draw [thick, fill=gray]  (8,0.5) rectangle (10,1.5);              
          %\draw [line width=1ex, --] (10,1) -- (11,1);                        
          \draw [thick, fill=gray]  (11,0.5) rectangle (13,1.5);             
          \node[text=gray!3] at (12,0.95)          {\Large{\textbf{TOS}}};                  
          %\draw [line width=1ex, --] (13,1)  -- (14.5,1);                       
          \draw [rounded corners, line width=1ex, <-] (6,0.5) -- (6,0) -- (12,0) -- (12,0.5);
        \end{scope}

      \end{tikzpicture}
    }} &
    %\multicolumn{1}{m{28em}|}{
    \makecell[l]{
    \begin{minipage}[t]{\linewidth}%  
      For all other stack operations, which use any of the \gls{rotext} paths and for which the
      ``\texttt{TUCK}'' rule does not apply,
      the \gls{ps} must hold at least \textbf{three} values prior to the stack operation.
    \end{minipage}%
    } \\ \hline
    
  \end{longtable}
\end{center}  
\endgroup
