%###############################################################################
%# N1 - Manual - Overview                                                      #
%###############################################################################
%#    Copyright 2018 Dirk Heisswolf                                            #
%#    This file is part of the N1 project.                                     #
%#                                                                             #
%#    N1 is free software: you can redistribute it and/or modify               #
%#    it under the terms of the GNU General Public License as published by     #
%#    the Free Software Foundation, either version 3 of the License, or        #
%#    (at your option) any later version.                                      #
%#                                                                             #
%#    N1 is distributed in the hope that it will be useful,                    #
%#    but WITHOUT ANY WARRANTY; without even the implied warranty of           #
%#    MERCHANTABILITY or FITNESS FOR A PARTICULAR PURPOSE.  See the            #
%#    GNU General Public License for more details.                             #
%#                                                                             #
%#    You should have received a copy of the GNU General Public License        #
%#    along with N1.  If not, see <http:%www.gnu.org/licenses/>.               #
%###############################################################################
%# Version History:                                                            #
%#   Novemmber 26, 2018                                                        #
%#      - Initial release                                                      #
%###############################################################################

\section{Overview}
\label{overview}

The N1 is a 16-bit stack machine, targeted for low-end FPGA applications.
It's instruction set and architecture is designed for efficient execution
of \Gls{forth} code. 

\noindent
Here is a summary of the N1's characteristics:

\begin{itemize}






  
%Stacks  
\item Hardware support for parameter and return stack 
  \begin{itemize}
  \item Cells are 16 bit wide
  \item Over and underflows monitored in hardware
  \item Upper cells of both stacks stored in registers
    \begin{itemize}
    \item Number of register based cells configurable via integration parameters
    \item Shift direction of topmost cells individually controllable, supporting
      common stack operations without additional data paths in hardware
    \item Data movement to and from RAM with hysteresis behavior to reduce bus accesses
    \end{itemize}
  \item Lower cells stored in RAM
    \begin{itemize}
    \item Dedicated address space for stacks, up to 128KB in size
    \item RAM space for stacks configurable via integration parameter
    \item Stacks allocated to grow towards eachother
    \item Dedicated whshbone interface for stack accesses
    \end{itemize}
  \end{itemize}
  
 %COF     






  
  
\end{itemize}

